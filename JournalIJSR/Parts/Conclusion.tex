This paper presented an experiment made to understand the contribution of angular and linear velocity, body's orientation, and movement direction on humans' perception of emotion expression by a non-anthropomorphic platform. The emotions covered in this experiment correspond to four of the ones enlisted by Ekman~\cite{Ekman2001} as basic emotions: anger, happiness, fear and sadness. To study the contribution of the desired features, a non-anthropomorphic, holonomic platform was used. The experiment was conducted at Politecnico di Milano, and involved 49 participants, each exposed to 20 over 195 movements, selected randomly for each participant. The Krippendorff's alpha agreement~\cite{Krippendorff2007} was used to calculate the consensus about the interpretation of each treatment. Using  alpha and the average emotion intensity attributed to each movement a top 10 movement table for each of the emotions was created.

The values obtained in this experiment could be used to express emotions in non-anthropomorphic platforms. It is still needed to cross validate the values and to determine what values from the top ten could more accurately express the desired emotion. It is important to mention that is not expected to have a 100\% of correct emotion recognition by the participants. As it has been observed in previous work in humans and robots, it is not possible to get a 100\% of correct recognition. This cross validation would help to reduce the number of feature combinations to the ones with higher possibility of identification. Moreover, additional experiments should be done to obtain analogous values for changes in shape and in different platform's size. 
