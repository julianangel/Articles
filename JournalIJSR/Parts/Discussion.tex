From the resulting tables~\ref{table:happy_top_ten},~\ref{table:Angry_top_ten},~\ref{table:Sad_top_ten}, and~\ref{table:Scared_top_ten}, it is possible to notice:

\begin{itemize}

	\item \textit{Fear} was the only emotion that had six over ten movements obtaining both general and specific alpha agreement over 0.41, which is the lower bound for moderate agreement~\cite{Viera2005}. Also, it was mostly selected (eight over ten) when $direction = \pi$ and $orientation = \pi$. The angle and the angular velocity attributed six over ten times were $0.087$ and $2$, respectively. So it seems that people perceive the movement as a fear expression when the robot is looking at them and moving far from them fast. 
	
	\item \textit{Sadness} was mostly attributed to linear velocities of $0 mm/s$ (two over ten) and $200 mm/s$ (eight over ten), and angular velocities of $0 rad/s$ and $1 rad/s$, with higher predilection of the second (eight over ten). It seems that people attribute sadness to slow velocities with slow angular velocity and small oscillation angle. Regarding the other two features, there is not any defined pattern that could lead to make a generalization. 

	\item \textit{Happiness} is attributed to different values of the independent variables. But it is the only emotion that is highly perceived when the linear velocity is $0$ with four over ten. However the oscillation angle for these four cases is equally divided between the values $0.174$ and $3.49$. %TODO BTW, is this angle in radians? It is a quite big difference...
So it seems that happiness is mainly attributed to fast angular velocities and big oscillations angles. Specific agreement among the other features is not present.

	\item \textit{Anger} is highly perceived with an angle of $0.087$ (seven over ten) and with linear velocities over $200 mm/s$. It seems that people attribute anger to fast velocities, both angular and linear, small angle of oscillation and the robot facing the persons while approaching them. 
\end{itemize} 
%alpha general
Although this information gives suggest some parameters that could be used to express the emotions studied, this suggestions are limited to the platform used on it. Hence, it is still necessary to study the influence that plays the embodiment on the perception of emotions. This is going to be beneficial to validate this study but it would also generalize other works done on the same direction.
