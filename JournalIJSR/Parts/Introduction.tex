%TODO Here, even before what follows, the need for emotion expression should be justified, ad possibily supported by literature

Emotions and mental states are not just presented through facial expression, but also by body postures and other features~\cite{Gelder2008}. However,  many psychological studies have been mostly focused on understanding the role of human face in emotion projection~\cite{Ekman2004},~\cite{kleinsmith2012affective} and mental states. This trend has been followed by the robotics community, where anthropomorphic faces (e.g.,~\cite{Arras2012},~\cite{Breazeal2002}) and bodies (e.g.,~\cite{Canamero2010},~\cite{haering2011},~\cite{Destephe2013}) have been widely used to convey emotions.
Nevertheless, in many situations the presence of anthropomorphic elements would be out of place and not justified by the main robot's functionalities. Most of the current and future robotics platforms on the market will not require anthropomorphic faces~\cite{Breazeal2002} or limbs~\cite{Li2011}. In some cases, like, for instance, in floor cleaning robots, anthropomorphic characteristics could even be detrimental to robot's task accomplishment.
This generates the necessity to study other mechanisms that could help to project emotions, which could give people an idea about the robots' state, and engage the user in long term relations.

The amount of works studying non-anthropomorphic features in robotics (e.g.,~\cite{Saerbeck2010,Lakatos2014,Sharma2013,Novika2015}) remains still small in comparison to those that exploit anthropomorphic features. Moreover, these works do not prescribe any specific range of values for the characteristics to be used to express the  implemented emotionsb. For example, Suk and collaborators~\cite{NAM2014}, in their study, give specific values for acceleration, curvature, but their connection to specific emotions is not given. Rather, they give a relationship between their features and values in terms of valence and arousal. This could lead to select features to convey certain emotions, but it would be better to know the specific range of values for those features to convey the single emotions, or to know the values that could lead to misinterpret the emotional expression.
 
In order to get a better understanding of other features and values that could be used to convey specific emotions, this paper presents an experiment that was designed to identify specific values for some movement features that could be used to express the following emotions, selected among the ones suggested by Ekman as basic~\cite{Ekman2004}: happiness, anger, fear, and sadness. The considered features are: oscillation angle, linear and angular velocity, direction and orientation, identified as independent variables. The perceived emotions and their intensities were considered as dependent variables. It is to be observed that it was given the possibility to the subjects to provide their evaluation of emotional intensity for more than one emotion for each experience. The experiment took place in Politecnico di Milano during June and July of 2015, where students from diverse departments were asked to participate without any economical retribution. Krippendorff's alpha agreement~\cite{Krippendorff2007} ($\alpha$) was used to evaluate the agreement among the participants for each treatment. For each of the four emotions was generated a top 10 list based on alpha agreement and perceived intensity. The results suggest that fear is perceived when the robot is looking at the subjects while moving far from them fast. Sadness is associated to slow velocities with slow angular velocity and small oscillation angle. Anger is attributed to fast velocities, both angular and linear, small angle of oscillation and the robot facing the subjects while approaching them.

This paper is organized as follows. The next section introduces previous studies done in human emotion and robot emotions projection. Section~\ref{sec:system} introduces the robotic platform and the software used in the experiment. The experiment's design is explained in section~\ref{sec:experiment}. Finally sections~\ref{sec:study} and~\ref{sec:result} present the study done and the obtained results.