There are many theories of emotion, differing in assumptions and components involved in the process. They can be classified in different ways. For example~\cite{Scherer2010,marsella_computational_2010} use the following categories:
\begin{itemize}
	\item \textit{Adaptational:} based on the idea that emotions are an evolving system used to detect stimuli that are of vital importance.
	\item \textit{Dimensional:} organize emotions according to different characteristics, usually valence (pleasantness-unpleasantness) and arousal. One of the most widely used is the Russel's circumplex model of affect~\cite{Russell1980}, which assumes that emotions can be mapped in a two dimensional plane. Two axis are used on this plane, one is the arousal (i.e. High and Low) and the other valance (i.e. pleasant and unpleasant). Other authors, such as Russel and Mehrabian~\cite{mehrabian1974approach} have suggested three axis to mapped emotions. These axis are pleasure, arousal and dominance, which is used to give the name to this model PAD. 
	\item \textit{Appraisal:} argues that emotions arise from the individual's judgment, based on its believes, desires, and intentions with respect to the current situation. This is a predominant theory among the others primarily because it gives explicit the link between emotion elicitation and response. 
	\item \textit{Motivational:} studies how motivational drives could generate emotions. Some theories then argued that emotions is a result of an evolutionary needs that should be cover.
	\item \textit{Circuit:} supports the fact that emotions correspond to a specific neuron path in the brain. Model that falls in this theory use the assumption that emotions' differentiation and the number of emotions are genetically coded as neural circuits.
	\item \textit{Discrete:} are theories based on Darwin's work, \textit{the expression of emotion in man and animals}. These theories use as a pillar the idea of the existence of a basic emotions that are international recognized. One of its most world known theorist is Paul Ekman, who has postulate the existence of six basic emotions (anger, fear, joy, sadness, and disgust) that are ''cross-cultural''~\cite{Ekman2004}. Other authors diverge in the amount of basic emotions and the ones that are included. Nevertheless, authors have started to integrated this theory with other theories. For example Izard~\cite{Izard2007} suggests two terms natural kinds and emotion schemas. Natural kinds are the basic emotions, which are emotions that are detected by specific parts of the brain and do not involve any kind of cognition. The emotion schemas involve cognition and may involve complex appraisal process and they would be what people call emotions.
	\item \textit{Other approaches} are lexical, social constructivist, anatomic, rational, and communicative.
\end{itemize}
In practice, these theoretical categories overlap. The difference among these theories is mainly in how the process, inputs and elicitation phases are considered in each one. This overlap goes along with the suggestion done by Izard~\cite{Izard1993}, among others, that emotion elicitation can be performed at different levels, and at some of these levels people are not aware of the process. 