%%%%%%%%%%%%%%%%%%%%%%% file template.tex %%%%%%%%%%%%%%%%%%%%%%%%%
%
% This is a general template file for the LaTeX package SVJour3
% for Springer journals.          Springer Heidelberg 2010/09/16
%
% Copy it to a new file with a new name and use it as the basis
% for your article. Delete % signs as needed.
%
% This template includes a few options for different layouts and
% content for various journals. Please consult a previous issue of
% your journal as needed.
%
%%%%%%%%%%%%%%%%%%%%%%%%%%%%%%%%%%%%%%%%%%%%%%%%%%%%%%%%%%%%%%%%%%%
%
% First comes an example EPS file -- just ignore it and
% proceed on the \documentclass line
% your LaTeX will extract the file if required
%\begin{filecontents*}{example.eps}
%%!PS-Adobe-3.0 EPSF-3.0
%%%BoundingBox: 19 19 221 221
%%%CreationDate: Mon Sep 29 1997
%%%Creator: programmed by hand (JK)
%%%EndComments
%gsave
%newpath
%  20 20 moveto
%  20 220 lineto
%  220 220 lineto
%  220 20 lineto
%closepath
%2 setlinewidth
%gsave
%  .4 setgray fill
%grestore
%stroke
%grestore
%\end{filecontents*}
%
\RequirePackage{fix-cm}
%
%\documentclass{svjour3}                     % onecolumn (standard format)
%\documentclass[smallcondensed]{svjour3}     % onecolumn (ditto)
%\documentclass[smallextended]{svjour3}       % onecolumn (second format)
\documentclass[twocolumn]{svjour3}          % twocolumn
%
\smartqed  % flush right qed marks, e.g. at end of proof
%
\usepackage{graphicx}
\usepackage{slashbox}     
\usepackage{subfigure}
\usepackage{breakcites}

%
% \usepackage{mathptmx}      % use Times fonts if available on your TeX system
%
% insert here the call for the packages your document requires
%\usepackage{latexsym}
% etc.
%
% please place your own definitions here and don't use \def but
% \newcommand{}{}
%
% Insert the name of "your journal" with
% \journalname{myjournal}
%
\begin{document}

\title{Identifying Values to Express Emotions with a Non-Anthropomorphic Platform%\thanks{Grants or other notes
%about the article that should go on the front page should be
%placed here. General acknowledgments should be placed at the end of the article.}
}
%\subtitle{Do you have a subtitle?\\ If so, write it here}

%\titlerunning{Short form of title}        % if too long for running head

\author{Julian M. Angel-Fernandez         \and
        Andrea Bonarini %etc.
}

%\authorrunning{Short form of author list} % if too long for running head

\institute{Julian M. Angel-Fernandez \at
              Automation and Control Institute, Vienna University of Technology,\\ Karlsplatz 13, 1040 Vienna, Austria \\
              Tel.: +43-158801\\
              \email{julian.angel.fernandez@tuwien.ac.at}           %  \\
%             \emph{Present address:} of F. Author  %  if needed
           \and
           Andrea Bonarini \at
              Andrea Bonarini, Dipartimento di Elettronica, Informazione e Bioingegneria,Politecnico di Milano,\\
              Piazza Leonardo da Vinci 32, 20133 Milan, Italy \\
              \email{andrea.bonarini@polimi.it}
}

\date{Received: date / Accepted: date}
% The correct dates will be entered by the editor


\maketitle

\begin{abstract}
Social environments are intricate spaces where people behave in ways that are socially accepted. In a similar way, robots that interact with humans on a daily bases are expected to develop certain behaviours to be socially accepted by people. This creates the necessity to point out our attention on social presence of robots, to let humans feel comfortable with them. The straightest way to fulfill this is through imitation of humans' characteristics, such us body shape, behaviors and social characteristics. As a consequence many social robots have been built with anthropomorphic embodiment, which could include features to express emotions. Although anthropomorphic embodiment may support robot acceptance in interactive tasks, there are robots that not required anthropomorphic shape due to the nature of the tasks that they have to perform. As a consequence,  features not related with anthropomorphic embodiment should be exploited to support robot acceptance. This paper presents an experiment made to understand humans' perception of emotion given angular and linear velocity, body orientation, and movement direction. Therefore, a non-anthropomorphic platform was used in the experiments to let participants to focus on the desired features. Moreover a Likert questionnaire was used to measure participants' perception for different treatments. The Krippendorff's alpha agreement was used to create a top 10 table for each of the emotions listed in the questionnaire. The obtained results suggest values that could guide the implementation of the considered emotions in social robots.
\keywords{Human-Robot Interaction \and Emotion Expression \and Experiment \and Non-anthropomorphic Platform \and Emotions Expression with Non-anthropomorphic Platform}
% \PACS{PACS code1 \and PACS code2 \and more}
% \subclass{MSC code1 \and MSC code2 \and more}
\end{abstract}

\section{Introduction}
Social environments are intricate spaces where people behave in ways that aim to their acceptance in social groups. Similarly, studies in social robotics have shown that robots acceptance increases when robots project a high social presence~\cite{Heerink08}. The straightest way to fulfill is through imitation of humans' characteristics, such us body form, behaviors and social characteristics. This has inspired most researchers to focus on how express emotions and mental states exploiting anthropomorphic features, in many occasions relying just on faces. However emotions and mental states are not just presented through facial expression, but also by body postures and other features~\cite{Gelder2008}. Nonetheless in psychology has been a clear tendency to study the role of human face in emotion projection~\cite{Ekman2004},~\cite{kleinsmith2012affective} and mental states. This trend has been followed by robotics community, where anthropomorphic faces (e.g.,~\cite{Arras2012},~\cite{Breazeal2002}) and bodies (e.g.,~\cite{Canamero2010},~\cite{haering2011},~\cite{Destephe2013}) have been widely used to convey emotions.
Nevertheless, in many situations the presence of anthropomorphic elements would be out of place and not justified by the main robot's functionalities. Most of the current and future robotics platforms on the market will not require anthropomorphic faces or limbs. In some cases, like, for instance, in floor cleaning robots, anthropomorphic characteristics could even be detrimental to robot's task accomplishment.
This generates the necessity to study other mechanisms that could help to project emotions, which could give people an idea about the robots' state, and engage the user in long term relations.

The amount of works studying non-anthropomorphic features in robotics (e.g.,~\cite{Saerbeck2010,Lakatos2014,Sharma2013,Novika2015}) remains still small in comparison to those that exploit anthropomorphic features. Moreover, these works do not prescribe any specific range of values for the characteristics to be used to express the  implemented emotions. For example, Suk and collaborators~\cite{NAM2014}, in their study, give specific values for acceleration, curvature, but their connection to specific emotions is not given. Rather, they gave a relationship between their features and values in terms of valence and arousal. Another possibility is the use of professional human actors to study how they convey certain emotions. However, a direct mapping between humans and robots is not possible~\cite{Saerbeck2007,Canamero2010} due to robots' physical capabilities. In addition, the significance of the agreement obtained from the studies that used human actors is still on discussion~\cite{Russell2003}. As a consequence is required the necessity to determine to project emotions in non-anthropomorphic platforms.
 
In order to get a better understanding of other features and values that could be used to convey specific emotions, this paper presents an experiment that was designed to identify specific values for some movement features that could be used to express the following emotions, selected among the ones suggested by Ekman as basic~\cite{Ekman2004}: happiness, anger, fear, and sadness. The considered features are: oscillation angle, linear and angular velocity, direction and orientation, identified as independent variables. The perceived emotions and their intensities were considered as dependent variables. It is to be observed that it was given the possibility to the subjects to provide their evaluation of emotional intensity for more than one emotion for each movement of the robot or treatment. The experiment took place in Politecnico di Milano during June and July of 2015, where students from diverse departments were asked to participate without any economical retribution. Krippendorff's alpha agreement~\cite{Krippendorff2007} ($\alpha$) was used to evaluate the agreement among the participants for each treatment. For each of the four emotions was generated a top 10 list based on alpha agreement and perceived intensity. The results suggest that fear is perceived when the robot is looking at the subjects while moving far from them fast. Sadness is associated to slow velocities with slow angular velocity and small oscillation angle. Anger is attributed to fast velocities, both angular and linear, small angle of oscillation and the robot facing the subjects while approaching them.

This paper is organized as follows. The next section introduces previous studies done in human emotion and robot emotions projection. Section~\ref{sec:system} introduces the robotic platform and the software used in the experiment. The experiment's design is explained in section~\ref{sec:experiment}. Finally sections~\ref{sec:experiment} and~\ref{sec:result} present the study done and the obtained results.
\section{Emotional Models}
There are many theories of emotion, differing in assumptions and components involved in the process. They can be classified in different ways. For example~\cite{Scherer2010,marsella_computational_2010} use the following categories:
\begin{itemize}
	\item \textit{Adaptational:} based on the idea that emotions are an evolving system used to detect stimuli that are of vital importance.
	\item \textit{Dimensional:} organize emotions according to different characteristics, usually valence (pleasantness-unpleasantness) and arousal. One of the most widely used is the Russel's circumplex model of affect~\cite{Russell1980}, which assumes that emotions can be mapped in a two dimensional plane. Two axis are used on this plane, one is the arousal (i.e. High and Low) and the other valance (i.e. pleasant and unpleasant). Other authors, such as Russel and Mehrabian~\cite{mehrabian1974approach} have suggested three axis to mapped emotions. These axis are pleasure, arousal and dominance, which is used to give the name to this model PAD. 
	\item \textit{Appraisal:} argues that emotions arise from the individual's judgment, based on its believes, desires, and intentions with respect to the current situation. This is a predominant theory among the others primarily because it gives explicit the link between emotion elicitation and response. 
	\item \textit{Motivational:} studies how motivational drives could generate emotions. Some theories then argued that emotions is a result of an evolutionary needs that should be cover.
	\item \textit{Circuit:} supports the fact that emotions correspond to a specific neuron path in the brain. Model that falls in this theory use the assumption that emotions' differentiation and the number of emotions are genetically coded as neural circuits.
	\item \textit{Discrete:} are theories based on Darwin's work, \textit{the expression of emotion in man and animals}. These theories use as a pillar the idea of the existence of a basic emotions that are international recognized. One of its most world known theorist is Paul Ekman, who has postulate the existence of six basic emotions (anger, fear, joy, sadness, and disgust) that are ''cross-cultural''~\cite{Ekman2004}. Other authors diverge in the amount of basic emotions and the ones that are included. Nevertheless, authors have started to integrated this theory with other theories. For example Izard~\cite{Izard2007} suggests two terms natural kinds and emotion schemas. Natural kinds are the basic emotions, which are emotions that are detected by specific parts of the brain and do not involve any kind of cognition. The emotion schemas involve cognition and may involve complex appraisal process and they would be what people call emotions.
	\item \textit{Other approaches} are lexical, social constructivist, anatomic, rational, and communicative.
\end{itemize}
In practice, these theoretical categories overlap. The difference among these theories is mainly in how the process, inputs and elicitation phases are considered in each one. This overlap goes along with the suggestion done by Izard~\cite{Izard1993}, among others, that emotion elicitation can be performed at different levels, and at some of these levels people are not aware of the process. 
\section{Related Work}
The direct consequence of the abundance of works in face elicitation in humans is the amount of works done in Human-Robot Interaction (HRI) focused on faces. One of the most well-known expressive robots is Kismet~\cite{Breazeal2002}, a robotic face able to interact with people and to show emotions. The face had enough degrees of freedom to portray the basic emotions suggested by Ekman~\cite{Ekman2004} (\textit{happiness}, \textit{surprise}, \textit{anger}, \textit{disgust}, \textit{fear}, and \textit{sadness}), plus \textit{interest}.

Studies focused on robotics bodies can be partitioned in two groups: platforms that resemble humans or animals, and those that don't. Moreover, the implemented emotions vary among works, with a tendency to include happiness. Questionnaires to assess people's perception are often adopted in works that use anthropomorphic platforms~\cite{Canamero2010,Beck2010,Li2011,Destephe2013b,Arras2012,Brown2014}. A relevant exception is the work done by Lakatos and collaborators~\cite{Lakatos2014} who assessed the participant's perception based on interaction through a play. In the experiment, each participant had two balls (yellow and black) and he could play with the robot using one of this balls. One of the balls would trigger robot's happiness and the other sadness. 

Regarding the movement or pose design, three major approaches could be found. In the first approach, actors are asked to perform different walks conveying emotions~\cite{Destephe2013b}. These movements are later reproduced with the robot, as closely as possible. A second approach is to ask some subjects to design the movements~\cite{Li2011}. The last approach is the empirical approach, where researches come with poses and movements based on their own experience~\cite{Canamero2010,Beck2010,Arras2012,Brown2014}. One major finding from diverse works~\cite{Canamero2010,Beck2010,Brown2014} is that moving the head up improved the identification of pride, happiness, and excitement. While moving the head down improved the identification of anger and sadness.  

On the other hand, the works that use platforms with no resemblance to persons or animals could be sub-divided in three groups. A first group study the possibility to convey emotions with their platforms~\cite{Arras2012, Novika2015, BarakovaL10}. The second group focus on determining the contribution of different features in the perception of emotions~\cite{Saerbeck2010,Barakova2013, Sharma2013, NAM2014}.
This group use pre-defined procedures to assess the participants' perception, such as Self-Assessment Manikin (SAM)~\cite{Lang2008}, possibly combined  with other procedures such as Social Dominance Orientation (SDO)~\cite{pratto1994social} and PANAS~\cite{WatsonClarkTellegen88}. 

It is important to mention the use of Laban’s Efforts~\cite{Laban1968} in the selection of the parameters in few works~\cite{BarakovaL10, Sharma2013}. Besides the fact that these works report that it was possible to convey emotions with non-human- or animal-like platforms, a relevant finding was the identification of the relationship between acceleration and speed with the arousal axis in the circumplex model of affect~\cite{Petta2010}.

Table~\ref{table:comparison_work_emotion_projection} summarizes all the robotics studies already mentioned, highlighting the following characteristics:
\begin{itemize}

	\item \textit{Embodiment} tells if the robot is human-like, robot-like or neither of these two;

	\item \textit{Platform used} gives the name of the platform used on the study;

	\item The\textit{Locomotion} gives information about the capabilities of the platform., because, even having a human-like embodiment, it does not mean that the platform can move;

	\item \textit{Emotion Evaluation Method} refers to the methodology used to collect the data in the studies.

	\item \textit{Emotion presentation} tells if a real robot was used in the study, or any other method;

	\item \textit{Emotions Implemented} gives the list of emotions showed to the subjects;
	
	\item \textit{How is the emotion conveyed?} This tells what medium was used to convey the emotion. The mediums considered were: movement, body posture and face poses.
	
\end{itemize}



\clearpage
\begin{table}[h]
\caption{Comparison among works on emotion projection in robotics. NA = Not Available}
\label{table:comparison_work_emotion_projection}
\begin{center}
\tiny %\begin{tabular}{|c|c|c|c|c|c|c|c|c|c|}
\begin{tabular}{|p{2.3 cm}|p{1.5 cm}|p{1.3 cm}|p{1.4 cm}|p{1.4 cm}|p{1.0 cm}|p{2.2 cm}|p{1.4 cm}|}
%\begin{tabular}{|p{1.5 cm}|p{1.0 cm}|p{1.5 cm}|p{1.5 cm}|p{1.5 cm}|p{2.0 cm}|p{1.5}|p{2.0 cm}|p{1.0 cm}|p{1.5 cm}|}
\hline 
\textbf{Work}  & \textbf{Embodiment} & \textbf{Platform Used} & \textbf{Locomotion} & \textbf{Emotion Evaluation Method} & \textbf{Emotion Presentation} & \textbf{Emotions Implemented}  & \textbf{How is the emotion conveyed?}\\ 
\hline 
Breazeal~\cite{Breazeal2002}  & Face & Kismet & NA & Questionnaire & Video & Happiness, surprise, anger, disgust, fear, sadness and interest &  Face poses\\ 
\hline
Saerbeck and Christoph~\cite{Saerbeck2010} & Animal / Non-Human like& iCat / Roomba & NA / Differential &PANAS and SAM & Real Robot & NA &  Movement\\
\hline 
Ca\~namero and collaborators~\cite{Canamero2010,Beck2010}&Human-Like & NAO & Bipedal & Questionnaire & Real robot & Anger, sadness, fear, pride, happiness, and excitement &  Body poses\\
\hline
Li and Chignell~\cite{Li2011}& Animal-Like& Teddy Bear Robot& NA & Questionnaire & Video (Real robot) & Random + Anger, disgust, fear, happiness, sadness, and surprise  & Body poses\\ 
\hline
Barakova and collaborators~\cite{Barakova2013} & NA & Closet robot& NA & SDO and SAM & Real robot & NA*  & Changing lights on and intensity\\ 
\hline
Sharma and collaborators~\cite{Sharma2013} & Non-Human / Animal & Quadrotor & Aerial & SAM + Questionnaire + Interview &  Real Robot & Laban's poles & Movement\\
\hline
Destephe and collaborators~\cite{Destephe2013b}&Human-like & WABIAN-2R & Bipedal & Questionnaire & Video (Virtual Robot) & Fear, anger, happiness, and sadness & Movement (Gait)+Body Poses\\
\hline 
Embgen and collaborators~\cite{Arras2012} & Human-like & Daryl &  Differential& Questionnaire & Real Robot & Happiness, sadness, fear, curiosity, embarrassment, and disappointment & Movement + Body poses \\ 
\hline
Lakatos and Collaborators~\cite{Lakatos2014}& Animal-Like & Dog & Holonomic & Indirect (Interaction with the robot) & Real Robot & Happiness and Fear &  Move + Body poses \\
\hline
Brown and Howard~\cite{Brown2014}& Human-Like & DARwIn-OP & Bipedal & Questionnaire & Real Robot & Happiness and sadness & Body poses\\
\hline
Novikova and Watss~\cite{Novika2015}&Non-Human/ Animal & Undetermine & Differential& Questionnaire& Real Robot & Scared, surprise, excited, angry, neutral, happiness and sadness &  Movement + Poses\\
\hline 
\end{tabular} 
\end{center}
\end{table}

Despite the possibility to use human actors to determine features and values to convey emotions or the findings done in the works previously discussed, there is a need to get a better understanding on how convey emotions with non-anthropomorphic platform. This is due to the following reasons. First, it is not possible to apply a direct mapping from human studies to robots~\cite{Saerbeck2007,Canamero2010} due to robots' physical capabilities. Second, the significance of the agreement obtained from the studies that used human actors is on discussion~\cite{Russell2003}. Third, no all possible features present in a non-anthropomorphic platform are studied in the literature (e.g. angular velocity). 
\section{The System}
\label{sec:system}

The system used in the experiment is composed by a non-anthropomorphic platform and software architecture that enrich robot's movements with emotion. The robotic platform was envisioned to be as simple as possible to limit the influence of shape on the study of the desired characteristics. 

%%%%%%%%%%%%%%%%%%%%%%%
\subsection{Hardware and Mechanical}

A holonomic platform was built to be used in the experiment. This type of platforms are characterized by the possibility to move in any direction without the necessity to have a specific orientation, i.e., they are free to move taking any desired orientation. Therefore, it was possible to imitate movements that are done by humans. For example, people do not have constaraints in movement, and can take any direction in any moment. The platform has a diameter of 25 cm and height of 25 cm. Figure~\ref{fig:ThirdDesign} shows the platform's blue prints and the real platform. Robot's frame of reference, in which all the velocities are going to be calculated, is depicted by the two black arrows. So, as it could be observed, to make the robot move forward a velocity along the $y$ axis is selected by the control system.

\begin{figure}
	\centering
	\includegraphics[width=0.48\textwidth]{Images/DesignThird.png} 
	\caption{Design of the platform. a) Base platform, this layer is used to carry the batteries. The two black arrows represent robot's frame of reference. b) First layer, which includes Arduino and the H-Bridges to control the motors. c) Second layer with Odroid-U3 and the mechanical structure to support the upper part. d) Lateral view of the robot.}
	\label{fig:ThirdDesign}
\end{figure}

%%%%%%%%%%%%%%%%%%%%%%%platform_base
\subsection{Software}
To guarantee that the correct robot's velocity for each motor could be achieved, a PID controller was implemented. PID's set point is established by a higher level controller that could receive two different types of commands. The first commands are robot's velocities ($<V_x,V_y,\omega>$), given in robot's reference frame. The second command is a set point ($<x, Y, \theta>$) to be reached, given in the world reference frame. This frame of reference is set every time the system is reset or boot. For example, if the robot finishes a trajectory and then it is reset, the new frame is going to be in the robot's current position.  %TODO From this description it seems that this second reference frame is again associated to the robot, in particular that it is a reference frame that is activated when the robot starts, and does not move with the robot. So it is a world reference frame reset each time the robot starts a new movement. It should have a name, to be found. If it was always associated with the robot it would have been called "deictic".

%TODO I'm sorry, but the following is an implementation detail that does not add anything to the scientific relevance of this paper, and this is not a paper about the interface.
%In addition to this low level control, a graphical interface, depicted in Figure~\ref{fig:experimental_interface}, was created to reduce the possibility of introducing wrong values for a desired sequence. This interface loads the sequences from a .txt file and displays sequences' numbers on it. Every time that a new sequence should be presented to a participant, the sequence's number is selected in the interface, which will display sequence's values. Once the robot has been positioned to the correct position, the execution could be started by clicking on send button. Here two commands are send, one resetting the controller and the second the desire position. In case that the sequence's execution should be aborted, it should be clicked the button stop.

%\begin{figure}
%	\centering
%	\includegraphics[width=0.45\textwidth]{./Images/ExperimentInterface.png} 
%	\caption{Interface used in the experiment. Once a sequence is selected, the interface shows sequence's values. Also, the interface give information about the current position of the robot and its velocity.}
%	\label{fig:experimental_interface}
%\end{figure}
\section{Experimental Design}
\label{sec:experiment}

The experiment was designed to get a better understanding about the contribution linear and angular velocity, oscillation angle, direction, and orientation can give to the expression of happiness, anger, sadness, and fear, which are four out of the six emotions considered by Ekman~\cite{Ekman2004} as basic emotions.
%%%%%%%%%%%%%%%%%%%
%%%%%%%%%%%%%%%%%%%

\subsection{Independent Variables}

The definitions of the selected independent variables are reported here below.

\begin{itemize}
	\item \textbf{Angular velocity} is the rotational velocity  ($\omega$) of the robot with respect to its center.

	\item \textbf{Oscillation angle} is the maximum extension of the robot's rotation around its center in the oscillating movement ($\theta$).

	\item \textbf{Linear velocity} is the rate of change of the position of the robot ($V$). 

	\item \textbf{Direction with respect to participant's perspective} is the angle generated from the participant's point of view with respect to the robot's trajectory ($D$).

	\item \textbf{Orientation of the body with respect to participant's perspective} is the robot's body orientation with respect to the robot's trajectory ($\phi$).

\end{itemize}

The three first variables are shown in Figure~\ref{fig:angular_movement}. The robot's frame of reference is drawn to show that it could move straight while it is rotating.


\begin{figure}
	\centering
	\includegraphics[width=0.48\textwidth]{./Images/ExampleMovement.png} 
	\caption{Example of the features used in the experiment. $x$ represents the displacement in meters, $\omega$ is the angular velocity ($rad/s$) and $\theta$ the oscillation of the body around its center ($rad$). The upper sequence depicts a movement based only on linear velocity, while the bottom one shows a sequence with angular and linear movement. The two black arrows in the robot's middle depict robot's frame of reference.}
	\label{fig:angular_movement}
	
\end{figure} 

\subsubsection*{Independent Variables values}

Specific, discrete values were selected for all independent variables, to make the experiment feasible. First, a simple test to evaluate when significant changes could be perceived was performed on a small sample of independent subjects. Based on this test, specific values for oscillation angle, and angular and linear velocities were selected. The values for the remaining two variables were selected according to what was needed by the experiment. %TODO I've tried to make this sentence clearer, but it is still obscure. What does it mean?

The chosen values are shown in Table~\ref{table:variables_values}. 

\begin{table}[htb]
\centering
\caption{Values for each of the independent variables in the experiment.}
\begin{tabular}{|c|c|c|c|c|}
\hline
\backslashbox{Variable}{Possibilities} & First & Second & Third & Fourth\\
\hline   
Angular Velocity ($rad/sec$)& $0$ & $1$ & $2$ & $3$\\
\hline
Oscillation Angle ($rad$)& $0$ & $0.087$ & $0.175$ & $0.349$\\
\hline
Linear Velocity ($mm/sec$) & $0$ & $200$ & $500$ & $900$\\
\hline
Direction ($rad$)&$0$&$\pi$&$\frac{-\pi}{2}$& \\
\hline
Orientation ($rad$) & $0$ & $\pi$ & & \\
\hline 
\multicolumn{5}{c}{}
\end{tabular} 
\label{table:variables_values}
\end{table}

To get a better idea about the Direction and Orientation values, in Figure~\ref{fig:possibilities_orientation_direction} are reported all the possibilities for these two variables.

\begin{figure}
	\centering
	\includegraphics[width=0.48\textwidth]{./Images/possibilities_case.png} 
	\caption{Combination of direction and orientation. The crosses symbolize the final position. The robot represents the initial position with its orientation, which is represented through the robot's frame of reference. The dashed big square represents the robot's movement zone, while the small one on the side represents the subjects' zone. a) Direction = $0$ and Orientation = $0$. b) Direction = $0$ and Orientation = $\pi$. c) Direction = $\pi$ and Orientation = $\pi$. d) Direction = $\pi$ and Orientation = $0$. e) Direction = $\frac{-\pi}{2}$ and Orientation = $0$. f) Direction = $\frac{-\pi}{2}$ and Orientation = $\pi$}
	\label{fig:possibilities_orientation_direction}
\end{figure}

The experiences, meant as desired procedures to compare~\cite{oehlert2000first}, %TODO What does "desired procedures to compare" mean?
were generated from the combination of independent variables' values for a total of 384 combinations. All the experiences that would not add any significant information to the experiment were removed, such as experiences with $\theta=0$ and $\omega \neq 0$, which reduced the total amount of experiences to 195.

%%%%%%%%%%%%%%%%%%%
%%%%%%%%%%%%%%%%%%%
%%%%%%%%%%%%%%%%%%%
\subsection{Dependent Variables}

\begin{itemize}
	\item \textbf{Emotion} is the feeling perceived by the participants from the robot's movement. From previous experiences %TODO cite
	, it was decided to ask the participants to select an emotion name in a list including the four emotions intended to be expressed, two mental states that could be misinterpreted from these emotions, and the option of ``other''. 
	The two  states of mind included as confounding terms are tenderness and excitement, which correspond to low and high arousal states.
	
	\item \textbf{Emotion's intensity} indicates the emotion intensity as perceived by the subject. This variable is measured on a ten point scale rate, ranging from 0 to 10, where 0 means that the corresponding emotion is not perceived by the subject and 10 that the emotion is highly perceived by the subject. 

\end{itemize}
%%%%%%%%%%%%%%%%%%%
%%%%%%%%%%%%%%%%%%%

%%%%%%%%%%%%%%%%%%%
\subsection{Participants}

It was decided that each subject had to be exposed to twenty over one hundred and ninety five possible experiences, trial that lasted  from 10 to 15 minutes. This was decided because each subject was a volunteer, so the time dedicated to the experiment had to be kept limited. The twenty treatments were selected picking a number without replacement from 1 to 195. A total of 980 answers were collected, guaranteeing a minimum of 5 trials for each motion type. 

The experiment was performed at Politecnico di Milano, involving subjects with different backgrounds.
A total of 49 volunteers were involved: 12 females and 37 males. The average age of the subjects was 25.28, with standard deviation of 2.8, with a minimum age of 20 and maximum of 32. The subjects' country of origin and their backgrounds are shown in the Table~\ref{table:country} and Table~\ref{table:career}, respectively. 

%%%%%%%%%%%%%%%%%%%%% 
\begin{table}[h]
\centering
\caption{Subjects' Country of origin.}
\label{table:country}
\begin{tabular}{|c|c|}
\hline
\textbf{Country} & \textbf{Counting} \\
\hline
Albania & 1 \\
\hline
Bosnia & 1 \\
\hline
Brazil & 2 \\
\hline
Colombia & 4 \\
\hline
Germany & 1 \\
\hline
Greece & 1 \\
\hline
Iran & 5 \\
\hline
Italy & 33 \\
\hline
Moldova & 1 \\
\hline 
\multicolumn{2}{c}{}
\end{tabular} 
\end{table}
%%%%%%%%%%%%%%%%

\begin{table}[h]
\centering
\caption{Subjects' Background.}
\label{table:career}
\begin{tabular}{|c|c|}
\hline
\textbf{Career} & \textbf{Counting}\\
\hline
Aeronautical Engineering & 1 \\
\hline
Architecture & 1 \\
\hline
Social assistance & 1 \\
\hline
Automation & 1 \\
\hline
Bio-medical Engineering & 5 \\
\hline
Computer Science & 33 \\
\hline
Electronic Engineering & 2 \\
\hline
Mechanical Engineering & 2 \\
\hline
Nursery & 1 \\
\hline
Pedagogical Science & 1 \\
\hline
Tourism & 1 \\
\hline
\multicolumn{2}{c}{}
\end{tabular} 
\end{table}
 %TODO I would say how many times each experience was tested. This is a needed parameter for this experimental setup.
%%%%%%%%%%%%%%%%%%%

The information from each subject was collected using a web-based form. To maintain anonymity, no personal information that could be used to trace the subject back was collected. The procedure used with each subject is described here below.

\begin{enumerate}
	
	\item The subject was asked to fill out the following information:
	
	\begin{itemize}
		\item Sex
		\item Background
		\item Age
		\item Country of origin
	\end{itemize}
	
	\item The robot was shown to the participant and the experiment procedure explained.

	\item An example of the questionnaire was presented and a sample movement of the robot shown.

	\item The subject was exposed to a specific movement sequence according to the following steps:

	\begin{enumerate}

		\item The subject is exposed to the movement generated by a configuration of values.

		\item The subject could use as much time as needed to select values for intensity of the different terms in the questionnaire.

		\item After the subject had completed his/her selection about the current movement, the robot is positioned to the new starting point, and the sequence is repeated from step 1.(a) for the rest of movements.

	\end{enumerate}
\end{enumerate}

The order of the options enlisted in the questionnaire changed for each question to prevent any kind of bias. Figure~\ref{fig:questionnaire_example} shows an example of the questionnaire used. 

\begin{figure}
	\centering
	\includegraphics[width=0.48\textwidth]{./Images/example_survey.png} 
	\caption{Example of the questionnaire used in the experiment.}
	\label{fig:questionnaire_example}
\end{figure}

\subsection{Setup}

The experimental setup and dimensions are shown in the Figure~\ref{fig:setup}. The crosses symbolize possible starting points, which were selected depending on the direction's value, as it is showed in Figure~\ref{fig:possibilities_orientation_direction}. 

\begin{figure}
	\centering
	\includegraphics[width=0.45\textwidth]{./Images/ExperimentGeneral.png} 
	\caption{Setup for the experiment. The crosses symbolize the possible starting points.}
	\label{fig:setup}
\end{figure} 
\section{Results}
\label{sec:result}
Table~\ref{table:result_fourth} summarizes the results obtained during the case study. 
\begin{table*}
\centering
\small
\caption{Summary of the answers obtained in the case study.}
		\label{table:result_fourth}
		\begin{tabular}{|c|c|c|c|c|c|c|c|c|c|c|c|c|c|}
			\hline	
			&\multicolumn{5}{|c|}{\textbf{Features}}&\multicolumn{7}{c|}{\textbf{Emotions}}&\\
			\cline{2-13}
\rotatebox{90}{\textbf{Presented/Reported } }&
\rotatebox{90}{\textbf{Direction  ($rad$)}}&
\rotatebox{90}{\textbf{Orientation ($rad$)} }&
\rotatebox{90}{\textbf{Linear Velocity ($mm/s$) }}&
\rotatebox{90}{\textbf{Angular Velocity ($rad/s$) }}&
\rotatebox{90}{\textbf{Angle ($rad$)}}&
\rotatebox{90}{\textbf{Happiness}}&
\rotatebox{90}{ \textbf{Anger}} &
\rotatebox{90}{\textbf{Fear}}&
\rotatebox{90}{\textbf{Sadness}}&
\rotatebox{90}{\textbf{Excitement}}&
\rotatebox{90}{\textbf{Tenderness}}&
\rotatebox{90}{\textbf{Other}}&
\rotatebox{90}{\textbf{Total}}\\	
			\hline
			\multirow{2}{*}{Happiness}&$0$&$0$&$500$&$3$&$0.349$&8&16&7&4&16&4&7&62\\
			\cline{2-14}
			&$0$&$0$&$900$&$3$&$0.174$&11&11&6&2&19&3&1&53\\
			\hline
			\multirow{2}{*}{Anger}&$\pi$&$0$&$500$&$3$&$0.087$&7&5&6&2&21&7&1&49\\
			\cline{2-14}
			&$0$&$0$&$900$&$1$&$0.087$&14&29&13&2&13&3&2&76\\
			\hline
			\multirow{2}{*}{Fear}&$\pi$&$\pi$&$900$&$2$&$0.174$&6&2&28&1&9&6&0&52\\
			\cline{2-14}
			&$\pi$&$\pi$&$500$&$2$&$0.087$&7&3&37&2&20&4&1&74\\
			\hline
			\multirow{2}{*}{Sadness}&$\pi$&$0$&$200$&$1$&$0.349$&3&5&17&14&5&16&5&65\\
			\cline{2-14}
			&$0$&$\pi$&$200$&$1$&$0.349$&5&5&15&28&6&15&7&81\\
			\hline
			\end{tabular}
\end{table*}

It could be observed that the two implementations of \textit{Happiness} were confused with \textit{Anger} and \textit{Excitement}. In a similar way, the first implementation of \textit{Anger} was mostly confused with \textit{Excitement}, which was voted twenty one over forty nine subjects that were exposed to the first implementation of \textit{Anger}.
The second implementation of \textit{Anger} showed an improvement of perception from 10\% to 38\%. This implementation was perceived also as \textit{Happiness}, \textit{Fear} and \textit{Excitement}.
Both implementations of \textit{Fear} had a high level of recognition 54 \% and 50 \% and mostly confused with \textit{Excitement}, which was voted nine times for the first implementation and twenty times for the second implementation. Finally, the two implementation of \textit{Sadness} was confused with \textit{Fear} and \textit{Tenderness}.

To verify these misinterpretations among the implemented emotions, a Fisher's exact test was applied for ten different combinations. Additionally, a Holm-Bonferroni correction was applied for multiple comparisons to get a better p-value estimation. The results are shown in Table~\ref{table:result_compare_fourth}. As this analysis suggest, the two implementation of \textit{Anger} were perceived as two different emotions. Also shows that the two implementation of Happiness were perceived to be similar to the second implementation of Anger.

\begin{table}
\centering
\small
\caption{Pair comparison among all the implemented emotions using Fisher's exact test for both questionnaires with $\alpha = 0.05$ for the  case study. The * indicates that the p-value was adjusted using the Holm-Bonferroni correction for multiple comparisons.}
		\label{table:result_compare_fourth}
		\begin{tabular}{|c|c|c|}
			\hline	
\textbf{Pair Compared} & \textbf{p-value} & \textbf{p-value*}\\	
			\hline
			Happiness 1 vs Happiness 2 &0.38&1.0\\
			\hline
			Anger 1 vs Anger 2 & 7.3e-4&4.4e-3\\
			\hline
			Anger 2 vs Happiness 1 & 0.137&0.69\\
			\hline
			Anger 2 vs Happiness 2 & 0.157&0.69\\
			\hline
			Fear 1 vs Fear 2 & 0.74&1.0\\
			\hline
			Sadness 1 vs Sadness 2 & 0.665&1.0\\
			\hline
			Fear 1 vs Sadness 1& 8.35e-5&5.8e-4\\
			\hline
			Fear 1 vs Sadness 2 & 5e-7&4e-6\\
			\hline
			Fear 2 vs Sadness 1 & 2e-7&1.8e-6\\
			\hline
			Fear 2 vs Sadness 2 & 1e-7&1e-6\\
			\hline
			\end{tabular}
\end{table} 

An analysis was done for each emotion, therefore it was created a contingency matrix such as was done in the previous studies. For each of these tables, the positive predictive value, accuracy and a Pearson's $\chi^2$ were computed. The results are shown in table~\ref{table:Precision2}. They show that there is significant evidence to conclude that second implementation of \textit{Anger}, both of \textit{Fear} and \textit{Sadness} have an impact in the perception of the emotion and they are considered as different implementation respect the rest of implementations. While both implementation of \textit{Happiness} and first of \textit{Anger} are considered as similar to the other implementation.

\clearpage
\begin{table*}
\centering
\caption{Accuracy, precision and results of Pearson's $\chi^2$ for each contingency matrix with $\alpha = 0.05$ for the case study.} 
\label{table:Precision2}
		\begin{tabular}{|p{3 cm}|p{2 cm}|c|c|c|}
		\hline
		\textbf{Presented Emotion} & \textbf{Positive Predicted Value} & \textbf{Accuracy} & \textbf{$\chi^2(1)$} & \textbf{p-value}\\
		\hline
		Happiness 1 & 0.13 & 0.79 & 0.11 & 0.74\\
		\hline
		Happiness 2 & 0.21 & 0.81& 3.7 &0.054\\
		\hline
		Anger 1 & 0.1 & 0.8 & 3.8e-29 & 1\\
		\hline
		Anger 2 & 0.38 & 0.81 & 34.4 & 4.47e-9\\
		\hline
		Fear 1 & 0.54 & 0.8 & 36.2 & 1.8-e9\\
		\hline 
		Fear 2 & 0.5 & 0.78 & 35.8 & 5.3e-10\\
		\hline
		Sadness 1 & 0.22 & 0.85 & 27.4 & 1.63e-7\\
		\hline
		Sadness 2 & 0.35 & 0.85 & 72.9 & 2.2e-16\\		 
		\hline
			\end{tabular}
\end{table*}  

 For each of the contingency tables the classification accuracy and the no-information rate (NIR), i.e. the accuracy that had been obtained by random selection, are reported in table~\ref{table:nir}. The results reveal that the only implementation with enough statistical evidence is the second implementation of \textit{Sadness}. 
Nevertheless, it is important no notice that the results were obtained using the lower part of the robot without any change in shape. Another important factor to highlight is the impact words enlisted in the questionnaire have on the perception rate. As it was expected in the experiment, \textit{Excitement} and \textit{Tenderness} were confused with other emotions with similar arousal level. In this precise case the emotions Anger and Happiness were confused with Excitement, and Sadness and Fear emotions were confused with Tenderness. Despite the bias generated by the two mental states enlisted in the questionnaire, the recognition rate of five out of eight implementations was over 35\%, being the two implementation of \textit{Fear} the implementations with the higher recognition rates (54\% for the first and 50\% for the second).

\begin{table*}
\centering
		\caption{Classification accuracy of the presented emotions by the single panels, computed as mentioned in the text, with corresponding 95\% confidence interval, no-information rate, and p-value that accuracy is greater than the NIR.}		
		\label{table:nir_fourth}
			\begin{tabular}{|c|c|c|c|c|c|c|c|c|c|}
				\hline
	& \multicolumn{5}{|c|}{\textbf{Features}} & & & &\\
\cline{2-6}				
\rotatebox{90}{\textbf{Presented Emotion}}&
\rotatebox{90}{\textbf{Direction  ($rad$)}}&
\rotatebox{90}{\textbf{Orientation ($rad$) }}&
\rotatebox{90}{\textbf{Linear Velocity ($mm/s$) }}&
\rotatebox{90}{\textbf{Angular Velocity ($rad/s$)} }&
\rotatebox{90}{\textbf{Angle ($rad$)}}&
\rotatebox{90}{\textbf{Classification Accuracy}}&
\rotatebox{90}{\textbf{95\% CI}}&
\rotatebox{90}{\textbf{No-Information Rate}}&
\rotatebox{90}{\textbf{P-Value [Acc $>$ NIR]}}\\
				\hline
			\multirow{2}{*}{Happiness}&$0$&$0$&$500$&$3$&$0.349$&0.79&(0.75,0.82)&0.89&1.0\\
			\cline{2-10}
			&$0$&$0$&$900$&$3$&$0.174$&0.81&(0.77,0.84)&0.88&1.0\\
			\hline
			\multirow{2}{*}{Anger}&$\pi$&$0$&$500$&$3$&$0.087$&0.8&(0.76,0.83)&0.88&1.0\\
			\cline{2-10}
			&$0$&$0$&$900$&$1$&$0.087$&0.89&(0.76,0.84)&0.83&0.95\\
			\hline
			\multirow{2}{*}{Fear}&$\pi$&$\pi$&$900$&$2$&$0.174$&0.79&(0.75,0.83)&0.88&1\\
			\cline{2-10}
			&$\pi$&$\pi$&$500$&$2$&$0.087$&0.78&(0.73,0.81)&0.83&0.99\\
			\hline
			\multirow{2}{*}{Sadness}&$\pi$&$0$&$200$&$1$&$0.349$&0.85&(0.81,0.88)&0.84&0.47\\
			\cline{2-10}
			&$0$&$\pi$&$200$&$1$&$0.349$&0.85&(0.81,0.88)&0.81&0.035\\
			\hline
			\end{tabular}
\end{table*}

The results obtained form the small scene are presented in Table~\ref{table:preference_selection}. A chi-squared test with one degree of freedom with an alpha of $0.5$ was done to verify if there was enough statistical evidence to accept our hypotheses: (i) people prefer scenes with emotions and (ii) gender has no impact on the preference. The results of the tests show that there is enough statistical evidence to accept our first hypothesis and reject the second one, with p-values of $1.42E-6$ and $0.85$, respectively.

Additionally, the Emotional Enrichment System was used in the two parts of the case used. Although, there was not done any measure of any variable of the system, two things could be said about the system. First it enables the possibility to adapt same script to different stage measures with any impact in the script.  Second, that it does not block the execution of an action when an emotion is changed.
\begin{table}
\centering
		\caption{Answers obtained for the small scene.}		
		\label{table:preference_selection}
			\begin{tabular}{|c|c|c|c|}
			\hline
			\textbf{Gender}&\textbf{With Emotion}&\textbf{Without Emotion}&\textbf{Total}\\
			\hline
			Male & 84 & 43 & 127\\
			\hline
			Female & 81 & 45 & 126\\
			\hline
			\end{tabular}
\end{table}

\section{Discussion}
\label{sec:discussion}
From the resulting tables~\ref{table:happy_top_ten},~\ref{table:Angry_top_ten},~\ref{table:Sad_top_ten}, and~\ref{table:Scared_top_ten}, it is possible to notice:

\begin{itemize}

	\item \textit{Fear} was the only emotion that had six over ten movements obtaining both general and specific alpha agreement over 0.41, which is the lower bound for moderate agreement~\cite{Viera2005}. Also, it was mostly selected (eight over ten) when $direction = \pi$ and $orientation = \pi$. The angle and the angular velocity attributed six over ten times were $0.087$ and $2$, respectively. So it seems that people perceive the movement as a fear expression when the robot is looking at them and moving far from them fast. 
	
	\item \textit{Sadness} was mostly attributed to linear velocities of $0 mm/s$ (two over ten) and $200 mm/s$ (eight over ten), and angular velocities of $0 rad/s$ and $1 rad/s$, with higher predilection of the second (eight over ten). It seems that people attribute sadness to slow velocities with slow angular velocity and small oscillation angle. Regarding the other two features, there is not any defined pattern that could lead to make a generalization. 

	\item \textit{Happiness} is attributed to different values of the independent variables. But it is the only emotion that is highly perceived when the linear velocity is $0$ with four over ten. However the oscillation angle for these four cases is equally divided between the values $0.174$ and $3.49$. %TODO BTW, is this angle in radians? It is a quite big difference...
So it seems that happiness is mainly attributed to fast angular velocities and big oscillations angles. Specific agreement among the other features is not present.

	\item \textit{Anger} is highly perceived with an angle of $0.087$ (seven over ten) and with linear velocities over $200 mm/s$. It seems that people attribute anger to fast velocities, both angular and linear, small angle of oscillation and the robot facing the persons while approaching them. 
\end{itemize} 
%alpha general
Although this information gives suggest some parameters that could be used to express the emotions studied, this suggestions are limited to the platform used on it. Hence, it is still necessary to study the influence that plays the embodiment on the perception of emotions. This is going to be beneficial to validate this study but it would also generalize other works done on the same direction.

\section{Conclusions and Further Work}
This paper presented an experiment made to understand the contribution of angular and linear velocity, body's orientation, and movement direction on humans' perception of emotion expression by a non-anthropomorphic platform. The emotions covered in this experiment correspond to four of the ones enlisted by Ekman~\cite{Ekman2001} as basic emotions: \textit{Anger}, \textit{Happiness}, \textit{Fear} and \textit{Sadness}. To study the contribution of the desired features, a non-anthropomorphic, holonomic platform was used. The experiment was conducted at Politecnico di Milano, and involved 49 participants, each exposed to 20 over 195 movements, selected randomly for each participant. The Krippendorff's alpha agreement~\cite{Krippendorff2007} was used to calculate the consensus about the interpretation of each treatment. Using  alpha and the average emotion intensity attributed to each movement a top 10 movement table for each of the emotions was created.

The values obtained in this experiment could be used as a guide to express emotions in non-anthropomorphic platforms with similar features as the one used in the experiment. It is still needed to cross validate the values and to determine what values from the top ten could more accurately express the desired emotion. It is important to mention that is not expected to have a 100\% of correct emotion recognition by the participants. As it has been observed in previous work in humans and robots, it is not possible to obtain a 100\% of correct recognition due to diverse factors such as current emotional state of the observer. This cross validation would help to reduce the number of feature combinations to the ones with higher possibility of identification. Moreover, additional experiments should be done to obtain analogous values for changes in shape and in different platform's size. Moreover, it is still necessary to study the influence that plays the embodiment on the perception of emotions. This is going to be beneficial to validate this study but it would also generalize other works done on the same direction.

Even thought the number of works studying how to project emotions with robotic platforms is increasing, it is still required a framework that standardize the processes for the design of this endevours. This could is due the following four reasongs. First, there are a variety of emotion theories that could be used as a based to design the experiments or case studies. Hence, researchers must know who to use them correctly and how they are connected among them. For example, the model suggested by Izard~\cite{Izard2007} redefines the idea of basic emotions and make a connection with dimensional theories. Second, researcher are using different platforms to do their experiments, at time of writing, there is not a formal evidence that clarify the impact of the embodiment on the perception of emotions. This generates situations in which two works could not be comparable. Third, the use of actors to generate the movements are not reliable. Actors use diverse techniques to create believable representations of specific emotions. However, this does not ensure that all actor convey emotions in same way. Therefore, several records are done and the ones with the highest agreement are selected. This brings the technical question on how to interpret the significance of agreement obtained~\cite{Russell2003}. Fourth, there is not a generic format to present the finding and movements used. Each researcher is using diverse methodologies to present their findings, for example some uses Laban's theory~\cite{Sharma2013} while others present the postures~\cite{NAO2013}.  
\bibliographystyle{spmpsci}
\bibliography{Bibliography,BibloNew,Biblography}

\end{document}
% end of file template.tex

