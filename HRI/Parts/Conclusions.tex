The case study presented in this paper was done to cross validate the findings in the experiment and verify whether the participants would prefer scenes when the robot expresses emotions or rather moves without any emotion expression. For each one of four emotions (i.e. Anger, Happiness, Sadness and Fear) studied in the experiment were selected a two set of parameters. These parameters were described in files and used as input to the Enrichment Emotional System, which was in charge to combine actions with emotions. The results show that both implementations of happiness were confused with anger and excitement, while one implementation of anger was just confused with excitement. Both implementations of sadness were confused with tenderness and fear. Both implementations of fear had a recognition rate over 50\%. Scene's results show that people prefer scenes with emotional movements and there is not any difference in gender.

Additionally to the results already mentioned, there are words that could bias participants' perception. For instance happiness and anger were considered as excitement. This misinterpretation should not be a surprise given the fact that there is not a unique definition of emotion~\cite{Plutchik2001,cacioppo2000handbook}, and each person would interpret a situation differently, so they will give a different label to the presented movement. Moreover a misinterpretation of Happiness and Anger could suggest that additional features (e.g. trajectory or shape) should be added to increase differentiation between these two emotions. For example, Venture and collaborators~\cite{Venture2014} had found out that in human bodies the recognition rate of anger and fear are increased when torso and head are downwards. On the other hand, they found that happiness perception is increased when the torso and head are move upwards. This example could bring some insight to possible body changes that could occur in non-human like bodies, but it should tested in this kind of platforms to confirm if the same impact is reached.
 