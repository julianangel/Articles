The case study presented in this chapter was done to cross validate the findings in the experiment. For each one of the four emotions studied in the experiment were selected two set of parameters. The results show that both implementations of happiness were confused with anger and excitement, while one implementation of anger was just confused with excitement. Both implementations of sadness were confused with tenderness and fear. Both implementations of fear had a recognition rate over 50\%. Additionally to the cross validation, it was done a small scene to check if people have preference to scenes where emotional movements are presented or not. The results show that people prefer scenes with emotional movements. 