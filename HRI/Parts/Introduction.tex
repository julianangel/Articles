Emotions and mental states are not just expressed through facial expression but also by body through postures and other features~\cite{Gelder2008}. Many psychological studies have been focused on understanding the role of human face in emotion projection~\cite{Ekman2004},~\cite{kleinsmith2012affective} and mental states. This trend has been followed by the robotics community, where anthropomorphic faces (e.g.,~\cite{Arras2012},~\cite{Breazeal2002}) and bodies (e.g.,~\cite{Canamero2010},~\cite{haering2011},~\cite{Destephe2013}) have been used to convey emotions.
However in many situations presence of anthropomorphic elements would be out of place and not justified by the main robot's functionalities. Most of the current and future robotics platforms on the market will not require anthropomorphic faces~\cite{Breazeal2002} or limbs~\cite{Li2011}, for instance in floor cleaning robots, anthropomorphic characteristics could even be detrimental to robot's task accomplishment.
This generates the necessity to study other mechanisms that could help to project emotions, which could give people an idea about the robots' state, and engage the user in long term relations.

It has been noticed that the amount of works studying non-anthropomorphic features in robotics (e.g.,~\cite{Saerbeck2010,Lakatos2014,Sharma2013,Novika2015}) remains still small in comparison to those exploiting anthropomorphic features. Moreover, these works do not give specific range of values for the characteristics used to expressed  emotions implemented. For example Suk and collaborators~\cite{NAM2014} in their study give specific values for acceleration and curvature,  but their connection to specific emotions is not explicited. Rather, they establish a relationship between their values and pleasure/arousal dimensions. This could help to select specific features to convey certain emotions, but it would be even better to know the precise range of values to be assigned to each characteristics to avoid  misinterpretation with wrong emotions. These values could be used in social robotics to show emotions and as consequence increase their acceptance.

This paper is organized as follows. Section~\ref{sec:related_work} introduce some previous works done in conveying emotion. Then the experiment is briefly described in section~\ref{sec:experiment}. Section~\ref{sec:system} presents the platform used in the case study and the Emotional Enrichment System. Finally, sections~\ref{sec:case} and~\ref{sec:results} presented case study's design and results.