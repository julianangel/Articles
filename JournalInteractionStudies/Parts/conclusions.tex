Four case studies were done to study the contribution of angular velocity, linear velocity, oscillation angle, and shape change in emotion expression from a non-bio-inspired robot. The results collected during all case studies show that a correct combination of linear and angular velocity, oscillation angle, and direction can be used to convey emotions that could be distinguished from each other. Also, they suggest that it is necessary to modify the robot's shape to increase people's emotion recognition rate. This result is congruent with the results obtained in the studies done to understand what features are more relevant when humans convey emotions~\cite{Roether2009,Venture2014}.

The lessons learned after these four case studies can be summarized as:

\begin{itemize}
	
	\item It is possible to convey emotions through changes in angular and linear velocity of a non-bio-inspired body also without changes in the body shape. This lesson opens the door to the inclusion of simple "emotion" to robotic platforms that are not required to have bio-inspired characteristics, but have to interact with humans.
	
	\item The change in the body shape increases the emotion identification rate of the observers. However, the characteristics that should be changed to increment the emotions perception are still to be investigated.
	
	\item Although Laban's notation is adequate to code people's movements, robots need precise values to define their actions. As a consequence, the Laban's theory needs to be instantiated to the robots' situation, to enable comparison between different works.
	
	\item Emotions come as a reaction to an event. Therefore, just showing uncorrelated movements to the participants is not going to support the study of the expression of all the emotions. For example, the action Fear was well perceived by the participants when the robot get far from them, as they were inducing it.
	
	\item Giving the participants the possibility to write their interpretation of the robot's movements was not going to be helpful to assess what emotion they thought the robot was conveying during case studies. Although there are plenty of procedures that have been proposed to determine the emotion that a person could be thinking at (e.g., SAM), they could just be used in a controlled environment where the experimenter could spend long periods of time with the participant. This is not the case of exhibitions where people would agree to stop just for a couple of minutes. On the other side, exhibitions make it possible to collect a large number of subjects representative of a varied population.
	
	\item Using a real platform to perform the case studies increases the enthusiasms of the participants and possibly their willingness to participate in the study. Although we did not evaluate directly this item during our case studies, it was observed during the set up that people were curious to see the robot moving. Also, during the pilot two kids had a different reaction when the robot was getting close to them. One of them was scared of the robot, while the other was curious about the robot.

\end{itemize}

Further work should be done to understand how the combination of changes in the body shape could contribute to the expression of emotions. It is also necessary to recreate a simple scene to give the context to the participants, but this time just using two robots. This will eliminate possible clues that could be given by the human actor.
