%%%%%%%%%%%%%%%%%%%%%%%%%%%%%%%%%%%%%%%%%%%%%%%%%%%%%%%%%%%%%%%%%%%%%%%%%%%%%%%%
%2345678901234567890123456789012345678901234567890123456789012345678901234567890
%        1         2         3         4         5         6         7         8

\documentclass[letterpaper, 10 pt, conference]{ieeeconf}  % Comment this line out if you need a4paper

%\documentclass[a4paper, 10pt, conference]{ieeeconf}      % Use this line for a4 paper
                                   % Needed to meet printer requirements.

% See the \addtolength command later in the file to balance the column lengths
% on the last page of the document

% The following packages can be found on http:\\www.ctan.org
\usepackage{graphics} % for pdf, bitmapped graphics files
\usepackage{epsfig} % for postscript graphics files
\usepackage{subfigure}
\usepackage{multirow} 
\usepackage{url}
\usepackage{slashbox}
\newcommand{\co}{\cellcolor{gray!40}}
\usepackage[table]{xcolor}
%\usepackage{mathptmx} % assumes new font selection scheme installed
%\usepackage{times} % assumes new font selection scheme installed
%\usepackage{amsmath} % assumes amsmath package installed
%\usepackage{amssymb}  % assumes amsmath package installed
\IEEEoverridecommandlockouts                              % This command is only needed if 
                                                          % you want to use the \thanks command

\overrideIEEEmargins   

\title{\LARGE \bf
Cross Validation of Emotional Features with a non-Anthropomorphic Platform
}


\author{Julian M. Angel-Fernandez$^{1}$ and Andrea Bonarini$^{2}$% <-this % stops a space
\thanks{$^{1}$Julian M. Angel-Fernandez is a Research at Automation and Control Institute, Vienna University of Technology, Vienna, Austria
        {\tt\small julian.angel.fernandez@tuwien.ac.at}}%
\thanks{$^{2}$Andrea Bonarini is a Professor at the Department of Electronics, Information, and Bioengineering, Politecnico di Milano, Milan, Italy,
        {\tt\small andrea.bonarini@polimi.it}}%
}


\begin{document}



\maketitle
\thispagestyle{empty}
\pagestyle{empty}


%%%%%%%%%%%%%%%%%%%%%%%%%%%%%%%%%%%%%%%%%%%%%%%%%%%%%%%%%%%%%%%%%%%%%%%%%%%%%%%%
\begin{abstract}
Robots should be able to express emotional states to interact with people as social agents. Emotions are a peculiar characterization of humans and usually conveyed through face expression or body language. However there are cases where robots cannot have anthropomorphic shape, for example when they have to perform tasks, which require a specific structure. This is the case of home cleaners, package carriers, and many others platforms. Therefore, emotional states need to be represented by exploiting other features, such as movements and shape changes. The work presented in this paper studies emotion expression in non-anthropomorphic platform and how it is perceived by humans. The work uses results obtained from a previous experiment, which was done to assess precise values for linear velocity, angular velocity, oscillation angle, direction and orientation that could be used to express happiness, angry, fear and sadness. Results show that people can recognize some emotional expressions better than others and that additional features should be added to differentiate some emotions. Additionally, it was done a small scene to verify if people prefer emotions with or without emotion. 
\end{abstract}

\section{Introduction}
Social environments are intricate spaces where people behave in ways that aim to their acceptance in social groups. Similarly, studies in social robotics have shown that robots acceptance increases when robots project a high social presence~\cite{Heerink08}. The straightest way to fulfill is through imitation of humans' characteristics, such us body form, behaviors and social characteristics. This has inspired most researchers to focus on how express emotions and mental states exploiting anthropomorphic features, in many occasions relying just on faces. However emotions and mental states are not just presented through facial expression, but also by body postures and other features~\cite{Gelder2008}. Nonetheless in psychology has been a clear tendency to study the role of human face in emotion projection~\cite{Ekman2004},~\cite{kleinsmith2012affective} and mental states. This trend has been followed by robotics community, where anthropomorphic faces (e.g.,~\cite{Arras2012},~\cite{Breazeal2002}) and bodies (e.g.,~\cite{Canamero2010},~\cite{haering2011},~\cite{Destephe2013}) have been widely used to convey emotions.
Nevertheless, in many situations the presence of anthropomorphic elements would be out of place and not justified by the main robot's functionalities. Most of the current and future robotics platforms on the market will not require anthropomorphic faces or limbs. In some cases, like, for instance, in floor cleaning robots, anthropomorphic characteristics could even be detrimental to robot's task accomplishment.
This generates the necessity to study other mechanisms that could help to project emotions, which could give people an idea about the robots' state, and engage the user in long term relations.

The amount of works studying non-anthropomorphic features in robotics (e.g.,~\cite{Saerbeck2010,Lakatos2014,Sharma2013,Novika2015}) remains still small in comparison to those that exploit anthropomorphic features. Moreover, these works do not prescribe any specific range of values for the characteristics to be used to express the  implemented emotions. For example, Suk and collaborators~\cite{NAM2014}, in their study, give specific values for acceleration, curvature, but their connection to specific emotions is not given. Rather, they gave a relationship between their features and values in terms of valence and arousal. Another possibility is the use of professional human actors to study how they convey certain emotions. However, a direct mapping between humans and robots is not possible~\cite{Saerbeck2007,Canamero2010} due to robots' physical capabilities. In addition, the significance of the agreement obtained from the studies that used human actors is still on discussion~\cite{Russell2003}. As a consequence is required the necessity to determine to project emotions in non-anthropomorphic platforms.
 
In order to get a better understanding of other features and values that could be used to convey specific emotions, this paper presents an experiment that was designed to identify specific values for some movement features that could be used to express the following emotions, selected among the ones suggested by Ekman as basic~\cite{Ekman2004}: happiness, anger, fear, and sadness. The considered features are: oscillation angle, linear and angular velocity, direction and orientation, identified as independent variables. The perceived emotions and their intensities were considered as dependent variables. It is to be observed that it was given the possibility to the subjects to provide their evaluation of emotional intensity for more than one emotion for each movement of the robot or treatment. The experiment took place in Politecnico di Milano during June and July of 2015, where students from diverse departments were asked to participate without any economical retribution. Krippendorff's alpha agreement~\cite{Krippendorff2007} ($\alpha$) was used to evaluate the agreement among the participants for each treatment. For each of the four emotions was generated a top 10 list based on alpha agreement and perceived intensity. The results suggest that fear is perceived when the robot is looking at the subjects while moving far from them fast. Sadness is associated to slow velocities with slow angular velocity and small oscillation angle. Anger is attributed to fast velocities, both angular and linear, small angle of oscillation and the robot facing the subjects while approaching them.

This paper is organized as follows. The next section introduces previous studies done in human emotion and robot emotions projection. Section~\ref{sec:system} introduces the robotic platform and the software used in the experiment. The experiment's design is explained in section~\ref{sec:experiment}. Finally sections~\ref{sec:experiment} and~\ref{sec:result} present the study done and the obtained results.
\label{sec:related_work}
\section{Related Work}
The use of emotion enrichment to improve human robot interaction is not new. There have been several researchers that have enhanced their social robots with emotions or studied how to convey emotions in robotic platforms~\cite{Li2011,Brown2014}. One of the first, well-known expressive robots is Kismet~\cite{Breazeal2002}, a robotic face able to interact with people and to show emotions. This platform uses a specific set of movements based on the Ekman's studies on human emotion expression~\cite{Ekman2004}. Other approaches have tried to use anthropomorphic~\cite{Arras2012} and human-like platforms to convey emotions to study the response of people towards the robot. However, the emotions portrayed were hand-coded, hard-wired to the respective platforms, and their parametrization is not available.

On the other hand, studies focused on entertainment robotics have tried to introduce emotional actions to improve the audience's experience. Breazeal and collaborators~\cite{Breazeal2003} used one robot on the stage. This anemone-like robot had few behaviours, which included getting scared when a person comes too close; the robot was able to show some basic emotions (i.e., fear and interest). Knight~\cite{Knight2011b,Knight2010} used the platform NAO to produce a sort of stand-up comedy. The robot performs basic actions to add some expressiveness to the joke, but it is not intended to project any emotion. Trying to add some theatrical realism, Breazeal and collaborators~\cite{Breazeal2008} designed and implemented a system to control a lamp. The main characteristic of this lamp is that it could be controlled by just one person, by selecting pre-coded emotions.

Other works in performance robotics have developed systems that do not convey any kind of emotion as \textit{Roboscopie}~\cite{Roboscopie2012,Lemaignan2012}, Fan and collaborators~\cite{Fan2009,Fan2013}, and adaptation of Shakespeare's  Midsummer Night's Dream~\cite{murphy2011} use robots in performances with real actors, but without any emotional expression.\\
Although these works use emotions, their main focus was the interpretation of postures or just the use of emotions to increase their robot appealing, but none of them considered the importance of an emotional system that could be used by others. As a consequence, most of these works have created emotional systems that could just work in their specific framework.

\section{System}
\label{sec:system}
A non-anthropomorphic robotic platform was created to do not have any bio-inspired appearance. The holonomic platform was built using Odroid U3, Arduino Due, and 3 metal gear motors with 64 CPR encoders and omniwheels. The platform could be observed in Figure~\ref{fig:Robot}. The Arduino Due is in charge to be the interface between the hardware (e.g. motors) and Odroid, which host all the Emotion Enrichment System.

\begin{figure}[t]
\centering%
\subfigure {\includegraphics[height=3cm]{./Images/DSC_0447.JPG}}
\hspace{2mm}
\subfigure{\includegraphics[height=3cm]{./Images/TriskarThird.png}}
\caption{Platform used in the case study (left), and holonomics's blue prints (right). The arrows represent robot's frame of reference.
\label{fig:Robot}}
\end{figure}
An Emotional Enrichment System was designed and implemented to automatize the process of emotion expression~\cite{Angel2017}. It modifies actions' parameters and adds additional actions to create the illusion of emotion expression in a robot. To achieve this, the system receives two messages. One message describes actions and the order in which they should be executed. These actions could be executed in parallel, sequentially or a combination of both. This kind of description enables the possibility to communicate several actions in one message. The other message informs the system with the desire emotion and its intensity. These two messages could arrive asynchronously and without any particular order. Every time a message is received, the system updates the robot movements to convey the desire emotion in the specific action.
\section{Case Study}
\label{sec:case}
The case study was done at Researcher's Night 2015 with two main objectives (i) cross-validate the findings obtained from the experiment, and (ii) use the Emotional Enrichment System to verify whether the participants would prefer scenes when the robot expresses emotions or rather moves without any emotion expression. The Emotional Enrichment System was used in both cases. To reduce scene variability was used two web-cams and eight Alvar tags to informed a Kalman filter to improve robots localization in the stage. The detection of the AR tags was done through the use of the ROS package ar\_track\_alvar~\cite{artag2015}. The distribution of the web-cams and the tags are depicted in Figure~\ref{fig:setup_fourth}. 

\begin{figure}
	\centering
	\includegraphics[width=0.45\textwidth]{./Images/FourthCase.png} 
	\caption{Environment setup for the case study.}
	\label{fig:setup_fourth}
\end{figure}

\subsection{Emotion Description}

The parameters selected for each of four emotions (\textit{Anger}, \textit{Happiness}, \textit{Sadness} and \textit{Fear}) are shown in Table~\ref{table:selected_fourth}. The main considerations to select the two implementations for each emotion were: (i) the linear velocity should be greater than $0$. In other words the robot should show some linear displacement. And (ii) it should be in the top 10 list of the emotion obtained in the experiment.

\begin{table}[h]
\centering
\small
\caption{Parameters' values selected from the experiment.}
		\label{table:selected_fourth}
		\begin{tabular}{|c|p{0.9 cm}|p{0.9 cm}|p{0.9 cm}|p{1.05 cm}|p{0.9 cm}|}
			\hline
%\rotatebox{90}{\textbf{Emotion } }&
%\rotatebox{90}{\textbf{Direction  ($rad$)}}&
%\rotatebox{90}{\textbf{Orientation ($rad$)} }&
%\rotatebox{90}{\textbf{Linear Velocity ($mm/s$) }}&
%\rotatebox{90}{\textbf{Angular Velocity ($rad/s$) }}&
%\rotatebox{90}{\textbf{Angle ($rad$)}}\\	
\textbf{Emotion}&\textbf{Direc-tion  ($rad$)} & \textbf{Orien-tation ($rad$)} & \textbf{Linear Velocity ($mm/s$) } & \textbf{Angular Velocity ($rad/s$) } & \textbf{Angle ($rad$)} \\
			\hline
			Happiness 1&$0$&$0$&$500$&$3$&$0.349$\\
			\hline
			Happiness 2&$0$&$0$&$900$&$3$&$0.174$\\
			\hline
			Anger 1&$\pi$&$0$&$500$&$3$&$0.087$\\
			\hline
			Anger 2&$0$&$0$&$900$&$1$&$0.087$\\
			\hline
			Fear 1&$\pi$&$\pi$&$900$&$2$&$0.174$\\
			\hline
			Fear 2&$\pi$&$\pi$&$500$&$2$&$0.087$\\
			\hline
			Sadness 1&$\pi$&$0$&$200$&$1$&$0.349$\\
			\hline
			Sadness 1&$0$&$\pi$&$200$&$1$&$0.349$\\
			\hline
			\end{tabular}
\end{table}


\subsection{Scene}

Stage discretization was used to give zones of movements instead of absolute positions. This idea was brought from human theatrical actors, who prepared their movements based on zones on the stage~\cite{wilson2009theatre}. This allows them to adapt their position based on other actors. Therefore, the stage was discretized in 9x9 matrix as is shown in Figure~\ref{fig:stage_division}. Robot's movements are given in terms of the matrix positions to the Emotional Enrichment System. This allows the adaptation to different stage dimensions because robot's final position is calculated by the Emotional Enrichment System during execution, which takes under consideration the stage dimensions given. For instance, during the scene's preparation in the laboratory the stage was 3 meters per 3 meters, but in the final presentation the stage was 2.5 meters per 2.5 meters.

\begin{figure}
	\centering
	\includegraphics[width=0.45\textwidth]{./Images/FourthCaseScene.png} 
	\caption{Stage discretization  used for the small scene. The blue squares correspond to the each zone, while the numbers correspond to the ID given to each zone.}
	\label{fig:stage_division}
\end{figure} 

The scene's description is the following: the robot starts in the middle of the stage to move to the upstage right (See~\cite{Musical}), close to the right wing. Then, the robot moves to upstage right center and rotates by $\pi/2$ left (See~\cite{Artopia}). Next the robot moves to the right center to then go to the center. When it arrives there, it turns full back and move backwards to downstage center with a full front orientation. There, it turns full back to move to center. Finally the robot turns to profile right and it does a step back; then it goes to the upstage center and then upstage right. The sequence of movements programmed to the robot are depicted in Figure~\ref{fig:movement}.
\begin{figure*}
	\centering
	\includegraphics[width=0.95\textwidth]{./Images/fourthCaseSceneD.png} 
	\caption{Sequence of movements done by the robot. The red arrows show the trajectory done by the robot, while the numbers show the order among the movements. a) The first ten movements b) The last five movements }
	\label{fig:movement}
\end{figure*}

The relation between emotion and movement is as follow: movements one, two, three, four and five are expressed without any emotion. Movements six, seven, eight, nine and ten show fear. Movement eleven depicts happiness, and the remaining movements depict sadness. The two scenes are executed by the Emotional Enrichment System using the same ''script'' and the emotion selection is done manually via graphical interface (Figure~\ref{fig:graphical_interface}).

\begin{figure}
	\centering
	\includegraphics[width=0.48\textwidth]{./Images/InterfaceExperiment.png} 
	\caption{Graphical interface used to communicate with the Emotional Enrichment System. a) It is the interface used to send the actions sequence. b) It is the interface used to send a emotion and its intensity.}
	\label{fig:graphical_interface}
\end{figure}

\subsection{Study}

This case study was done during Researchers' Night, 2015. During a period of two days, people were asked to participate to this study. Each subject was exposed to two rounds, in each one the robot was performing a different emotion. And  they were also exposed twice to a small scene, one with emotion and other without emotions. The emotions showed in each trial and the order of the scenes (with or without emotion) were generated randomly beforehand. The total number of volunteers was 256: 128 males, 126 females, and 2 that chose not to specify their gender. The average age was 27.29 years, with standard deviation of 16.58, minimum age was 4 and maximum 76.

\section{Results}
\label{sec:results}
Table~\ref{table:result_fourth} summarizes the results obtained during the case study. 
\begin{table*}
\centering
\small
\caption{Summary of the answers obtained in the case study.}
		\label{table:result_fourth}
		\begin{tabular}{|c|c|c|c|c|c|c|c|c|c|c|c|c|c|}
			\hline	
			&\multicolumn{5}{|c|}{\textbf{Features}}&\multicolumn{7}{c|}{\textbf{Emotions}}&\\
			\cline{2-13}
\rotatebox{90}{\textbf{Presented/Reported } }&
\rotatebox{90}{\textbf{Direction  ($rad$)}}&
\rotatebox{90}{\textbf{Orientation ($rad$)} }&
\rotatebox{90}{\textbf{Linear Velocity ($mm/s$) }}&
\rotatebox{90}{\textbf{Angular Velocity ($rad/s$) }}&
\rotatebox{90}{\textbf{Angle ($rad$)}}&
\rotatebox{90}{\textbf{Happiness}}&
\rotatebox{90}{ \textbf{Anger}} &
\rotatebox{90}{\textbf{Fear}}&
\rotatebox{90}{\textbf{Sadness}}&
\rotatebox{90}{\textbf{Excitement}}&
\rotatebox{90}{\textbf{Tenderness}}&
\rotatebox{90}{\textbf{Other}}&
\rotatebox{90}{\textbf{Total}}\\	
			\hline
			\multirow{2}{*}{Happiness}&$0$&$0$&$500$&$3$&$0.349$&8&16&7&4&16&4&7&62\\
			\cline{2-14}
			&$0$&$0$&$900$&$3$&$0.174$&11&11&6&2&19&3&1&53\\
			\hline
			\multirow{2}{*}{Anger}&$\pi$&$0$&$500$&$3$&$0.087$&7&5&6&2&21&7&1&49\\
			\cline{2-14}
			&$0$&$0$&$900$&$1$&$0.087$&14&29&13&2&13&3&2&76\\
			\hline
			\multirow{2}{*}{Fear}&$\pi$&$\pi$&$900$&$2$&$0.174$&6&2&28&1&9&6&0&52\\
			\cline{2-14}
			&$\pi$&$\pi$&$500$&$2$&$0.087$&7&3&37&2&20&4&1&74\\
			\hline
			\multirow{2}{*}{Sadness}&$\pi$&$0$&$200$&$1$&$0.349$&3&5&17&14&5&16&5&65\\
			\cline{2-14}
			&$0$&$\pi$&$200$&$1$&$0.349$&5&5&15&28&6&15&7&81\\
			\hline
			\end{tabular}
\end{table*}

It could be observed that the two implementations of \textit{Happiness} were confused with \textit{Anger} and \textit{Excitement}. In a similar way, the first implementation of \textit{Anger} was mostly confused with \textit{Excitement}, which was voted twenty one over forty nine subjects that were exposed to the first implementation of \textit{Anger}.
The second implementation of \textit{Anger} showed an improvement of perception from 10\% to 38\%. This implementation was perceived also as \textit{Happiness}, \textit{Fear} and \textit{Excitement}.
Both implementations of \textit{Fear} had a high level of recognition 54 \% and 50 \% and mostly confused with \textit{Excitement}, which was voted nine times for the first implementation and twenty times for the second implementation. Finally, the two implementation of \textit{Sadness} was confused with \textit{Fear} and \textit{Tenderness}.

To verify these misinterpretations among the implemented emotions, a Fisher's exact test was applied for ten different combinations. Additionally, a Holm-Bonferroni correction was applied for multiple comparisons to get a better p-value estimation. The results are shown in Table~\ref{table:result_compare_fourth}. As this analysis suggest, the two implementation of \textit{Anger} were perceived as two different emotions. Also shows that the two implementation of Happiness were perceived to be similar to the second implementation of Anger.

\begin{table}
\centering
\small
\caption{Pair comparison among all the implemented emotions using Fisher's exact test for both questionnaires with $\alpha = 0.05$ for the  case study. The * indicates that the p-value was adjusted using the Holm-Bonferroni correction for multiple comparisons.}
		\label{table:result_compare_fourth}
		\begin{tabular}{|c|c|c|}
			\hline	
\textbf{Pair Compared} & \textbf{p-value} & \textbf{p-value*}\\	
			\hline
			Happiness 1 vs Happiness 2 &0.38&1.0\\
			\hline
			Anger 1 vs Anger 2 & 7.3e-4&4.4e-3\\
			\hline
			Anger 2 vs Happiness 1 & 0.137&0.69\\
			\hline
			Anger 2 vs Happiness 2 & 0.157&0.69\\
			\hline
			Fear 1 vs Fear 2 & 0.74&1.0\\
			\hline
			Sadness 1 vs Sadness 2 & 0.665&1.0\\
			\hline
			Fear 1 vs Sadness 1& 8.35e-5&5.8e-4\\
			\hline
			Fear 1 vs Sadness 2 & 5e-7&4e-6\\
			\hline
			Fear 2 vs Sadness 1 & 2e-7&1.8e-6\\
			\hline
			Fear 2 vs Sadness 2 & 1e-7&1e-6\\
			\hline
			\end{tabular}
\end{table} 

An analysis was done for each emotion, therefore it was created a contingency matrix such as was done in the previous studies. For each of these tables, the positive predictive value, accuracy and a Pearson's $\chi^2$ were computed. The results are shown in table~\ref{table:Precision2}. They show that there is significant evidence to conclude that second implementation of \textit{Anger}, both of \textit{Fear} and \textit{Sadness} have an impact in the perception of the emotion and they are considered as different implementation respect the rest of implementations. While both implementation of \textit{Happiness} and first of \textit{Anger} are considered as similar to the other implementation.

\clearpage
\begin{table*}
\centering
\caption{Accuracy, precision and results of Pearson's $\chi^2$ for each contingency matrix with $\alpha = 0.05$ for the case study.} 
\label{table:Precision2}
		\begin{tabular}{|p{3 cm}|p{2 cm}|c|c|c|}
		\hline
		\textbf{Presented Emotion} & \textbf{Positive Predicted Value} & \textbf{Accuracy} & \textbf{$\chi^2(1)$} & \textbf{p-value}\\
		\hline
		Happiness 1 & 0.13 & 0.79 & 0.11 & 0.74\\
		\hline
		Happiness 2 & 0.21 & 0.81& 3.7 &0.054\\
		\hline
		Anger 1 & 0.1 & 0.8 & 3.8e-29 & 1\\
		\hline
		Anger 2 & 0.38 & 0.81 & 34.4 & 4.47e-9\\
		\hline
		Fear 1 & 0.54 & 0.8 & 36.2 & 1.8-e9\\
		\hline 
		Fear 2 & 0.5 & 0.78 & 35.8 & 5.3e-10\\
		\hline
		Sadness 1 & 0.22 & 0.85 & 27.4 & 1.63e-7\\
		\hline
		Sadness 2 & 0.35 & 0.85 & 72.9 & 2.2e-16\\		 
		\hline
			\end{tabular}
\end{table*}  

 For each of the contingency tables the classification accuracy and the no-information rate (NIR), i.e. the accuracy that had been obtained by random selection, are reported in table~\ref{table:nir}. The results reveal that the only implementation with enough statistical evidence is the second implementation of \textit{Sadness}. 
Nevertheless, it is important no notice that the results were obtained using the lower part of the robot without any change in shape. Another important factor to highlight is the impact words enlisted in the questionnaire have on the perception rate. As it was expected in the experiment, \textit{Excitement} and \textit{Tenderness} were confused with other emotions with similar arousal level. In this precise case the emotions Anger and Happiness were confused with Excitement, and Sadness and Fear emotions were confused with Tenderness. Despite the bias generated by the two mental states enlisted in the questionnaire, the recognition rate of five out of eight implementations was over 35\%, being the two implementation of \textit{Fear} the implementations with the higher recognition rates (54\% for the first and 50\% for the second).

\begin{table*}
\centering
		\caption{Classification accuracy of the presented emotions by the single panels, computed as mentioned in the text, with corresponding 95\% confidence interval, no-information rate, and p-value that accuracy is greater than the NIR.}		
		\label{table:nir_fourth}
			\begin{tabular}{|c|c|c|c|c|c|c|c|c|c|}
				\hline
	& \multicolumn{5}{|c|}{\textbf{Features}} & & & &\\
\cline{2-6}				
\rotatebox{90}{\textbf{Presented Emotion}}&
\rotatebox{90}{\textbf{Direction  ($rad$)}}&
\rotatebox{90}{\textbf{Orientation ($rad$) }}&
\rotatebox{90}{\textbf{Linear Velocity ($mm/s$) }}&
\rotatebox{90}{\textbf{Angular Velocity ($rad/s$)} }&
\rotatebox{90}{\textbf{Angle ($rad$)}}&
\rotatebox{90}{\textbf{Classification Accuracy}}&
\rotatebox{90}{\textbf{95\% CI}}&
\rotatebox{90}{\textbf{No-Information Rate}}&
\rotatebox{90}{\textbf{P-Value [Acc $>$ NIR]}}\\
				\hline
			\multirow{2}{*}{Happiness}&$0$&$0$&$500$&$3$&$0.349$&0.79&(0.75,0.82)&0.89&1.0\\
			\cline{2-10}
			&$0$&$0$&$900$&$3$&$0.174$&0.81&(0.77,0.84)&0.88&1.0\\
			\hline
			\multirow{2}{*}{Anger}&$\pi$&$0$&$500$&$3$&$0.087$&0.8&(0.76,0.83)&0.88&1.0\\
			\cline{2-10}
			&$0$&$0$&$900$&$1$&$0.087$&0.89&(0.76,0.84)&0.83&0.95\\
			\hline
			\multirow{2}{*}{Fear}&$\pi$&$\pi$&$900$&$2$&$0.174$&0.79&(0.75,0.83)&0.88&1\\
			\cline{2-10}
			&$\pi$&$\pi$&$500$&$2$&$0.087$&0.78&(0.73,0.81)&0.83&0.99\\
			\hline
			\multirow{2}{*}{Sadness}&$\pi$&$0$&$200$&$1$&$0.349$&0.85&(0.81,0.88)&0.84&0.47\\
			\cline{2-10}
			&$0$&$\pi$&$200$&$1$&$0.349$&0.85&(0.81,0.88)&0.81&0.035\\
			\hline
			\end{tabular}
\end{table*}

The results obtained form the small scene are presented in Table~\ref{table:preference_selection}. A chi-squared test with one degree of freedom with an alpha of $0.5$ was done to verify if there was enough statistical evidence to accept our hypotheses: (i) people prefer scenes with emotions and (ii) gender has no impact on the preference. The results of the tests show that there is enough statistical evidence to accept our first hypothesis and reject the second one, with p-values of $1.42E-6$ and $0.85$, respectively.

Additionally, the Emotional Enrichment System was used in the two parts of the case used. Although, there was not done any measure of any variable of the system, two things could be said about the system. First it enables the possibility to adapt same script to different stage measures with any impact in the script.  Second, that it does not block the execution of an action when an emotion is changed.
\begin{table}
\centering
		\caption{Answers obtained for the small scene.}		
		\label{table:preference_selection}
			\begin{tabular}{|c|c|c|c|}
			\hline
			\textbf{Gender}&\textbf{With Emotion}&\textbf{Without Emotion}&\textbf{Total}\\
			\hline
			Male & 84 & 43 & 127\\
			\hline
			Female & 81 & 45 & 126\\
			\hline
			\end{tabular}
\end{table}

\section{Conclusions and Further work}
The case study presented in this paper was done to cross validate the findings in the experiment and verify whether the participants would prefer scenes when the robot expresses emotions or rather moves without any emotion expression. For each one of four emotions (i.e. Anger, Happiness, Sadness and Fear) studied in the experiment were selected a two set of parameters. These parameters were described in files and used as input to the Enrichment Emotional System, which was in charge to combine actions with emotions. The results show that both implementations of happiness were confused with anger and excitement, while one implementation of anger was just confused with excitement. Both implementations of sadness were confused with tenderness and fear. Both implementations of fear had a recognition rate over 50\%. Scene's results show that people prefer scenes with emotional movements and there is not any difference in gender.

Additionally to the results already mentioned, there are words that could bias participants' perception. For instance happiness and anger were considered as excitement. This misinterpretation should not be a surprise given the fact that there is not a unique definition of emotion~\cite{Plutchik2001,cacioppo2000handbook}, and each person would interpret a situation differently, so they will give a different label to the presented movement. Moreover a misinterpretation of Happiness and Anger could suggest that additional features (e.g. trajectory or shape) should be added to increase differentiation between these two emotions. For example, Venture and collaborators~\cite{Venture2014} had found out that in human bodies the recognition rate of anger and fear are increased when torso and head are downwards. On the other hand, they found that happiness perception is increased when the torso and head are move upwards. This example could bring some insight to possible body changes that could occur in non-human like bodies, but it should tested in this kind of platforms to confirm if the same impact is reached.
 
\bibliographystyle{IEEEtran}
\bibliography{Bibliography,BibloNew,Biblography}

\addtolength{\textheight}{-12cm}
\end{document}
