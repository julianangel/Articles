The case study presented in this paper was done to cross validate the findings in a previous experiment~\cite{Angel2017-2}, and to verify whether the subjects would prefer scenes when the robot expresses emotions rather than the same trajectories without any emotion expression. For each one of four emotions (i.e., \textit{Anger}, \textit{Happiness}, \textit{Sadness} and \textit{Fear}) studied in the experiment two sets of parameters were selected to express them. The results show that both implementations of \textit{Happiness} were confused with \textit{Anger} and \textit{Excitement}, while one implementation of \textit{Anger} was confused with \textit{Excitement}. Both implementations of \textit{Sadness} were confused with \textit{Tenderness} and \textit{Fear}. This supports the hypothesis that the selected parameter values are related to arousal, since confusion appears among emotions with high arousal, and among those with low arousal. Both implementations of \textit{Fear} had a recognition rate over 50\%. This, together with the last statement may suggest that people have different models of \textit{Fear} and that this single label is not enough to distinguish among them. 

In the scene trials, people showed, without any difference in gender, to prefer scenes with emotional movements.

Additionally to the already mentioned results, it resulted that there are words that could bias participants' perception. For instance, \textit{Happiness} and \textit{Anger} were considered as \textit{Excitement}. This misinterpretation is not surprising, given the fact that there is not a unique definition of emotion~\cite{Plutchik2001,cacioppo2000handbook}, and each person would interpret a situation differently, so they will give a different label to the presented movement. Moreover, a misinterpretation of \textit{Happiness} and \textit{Anger} could suggest that additional features (e.g., trajectory or shape) should be added to increase differentiation between them. For example, Venture and collaborators~\cite{Venture2014} had reported that in human bodies the recognition rate of \textit{Anger} and \textit{Fear} are increased when torso and head are downwards. On the other hand, they found that \textit{Happiness} perception is increased when the torso and head are moved upwards. These results could bring some insight to possible body changes that could occur in non-human like bodies, but it should still be tested in this kind of platforms.
 
