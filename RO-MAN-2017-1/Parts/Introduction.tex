Emotions and mental states are not just expressed through facial expression and voice, but also by body postures and other features~\cite{Gelder2008}. Many psychological studies have focused their attention on how emotions are conveyed through the face~\cite{kleinsmith2012affective}. Similarly, the robotics community uses anthropomorphic faces (e.g.~\cite{Breazeal2002}) and bodies (e.g.~\cite{Canamero2010}) to convey emotions.
In many situations, the presence of anthropomorphic elements would be out of place and not justified by the main robot's functionalities. Most of the current and future robotics platforms on the market will not require anthropomorphic faces~\cite{Breazeal2002} or limbs~\cite{Li2011}. For instance, floor cleaning robots are robots that do not have any anthropomorphic characteristics and still they accomplish their task.
This generates the necessity to study other mechanisms that could help to convey emotions, which could give people an idea about the robot's state, and engage the user in long term relations.

However,the amount of works studying non-anthropomorphic features in robotics (e.g.,~\cite{Saerbeck2010}) remains still small in comparison to those exploiting anthropomorphic features. Moreover, these works do not provide specific range of values for the characteristics used to express the implemented emotions. For example Suk and collaborators~\cite{NAM2014} in their study give specific values for acceleration and curvature, but their connection to specific emotions is not described. Rather, they established a relationship between their values and pleasure/arousal dimensions. This could help to select specific features to convey certain emotions, but it would be better to know the precise range of values to be assigned to each characteristic to avoid misinterpretation with wrong emotions. These values could be used in social robotics to show emotions and, as consequence, increase their acceptance.

This paper is organized as follows. Section~\ref{sec:related_work} introduces some previous works done in conveying emotion. Section~\ref{sec:system} presents the platform used in the case study and the Emotional Enrichment System. Finally, sections~\ref{sec:case} and~\ref{sec:results} present case study's design and results.
