 The case study presented in this paper was held at an exhibition with two main objectives:
 (i) Cross-validate the findings obtained from a previous experiment~\cite{Angel2017-2}, which was done to study the attribution of different linear and angular velocities, oscillation angle, direction, and orientation of the platform to express: \textit{Anger}, \textit{Happiness}, \textit{Sadness} and \textit{Fear}. A total of 196 value combinations were designed, 20 of which, randomly selected, were presented to each participant. Subjects had to select for each presentation the most representative term describing it among the four emotions and two mental states (i.e., \textit{Excitement} and \textit{Tenderness}), added as confounding terms. (ii) Verify whether participants would prefer scenes where the robot moves expressing emotions, or rather moves on the same trajectory without any emotion expression. To control scene variability two cameras and eight AR tags were used together with a Kalman filter to improve robot localization on the stage. AR tags were detected using the ROS package ar\_track\_alvar~\cite{artag2015}. The distribution of the cameras and the tags is reported in Figure~\ref{fig:setup_fourth}.

\begin{figure}
	\centering
	\includegraphics[width=0.45\textwidth]{./Images/FourthCase.png} 
	\caption{Environment setup for the case study. The crosses represent the starting points.}
	\label{fig:setup_fourth}
\end{figure}

\subsection{Emotion Description}

The features selected to implement \textit{Anger}, \textit{Happiness}, \textit{Sadness} and \textit{Fear} are shown in Table~\ref{table:selected_fourth}. As it could be observed, two configurations for each emotion were selected among the ones evaluated in the previous pilot, to be cross-checked. These configurations had to satisfy two conditions: (i) the linear velocity had to be greater than $0$, so the robot could show some displacement, and (ii) it should be in the top 10 list of the configurations tested in the previous pilot experiment.

\begin{table}
\centering
\small
\caption{Parameter values selected from the previous experiment.}
		\label{table:selected_fourth}
		\begin{tabular}{|c|p{0.9 cm}|p{0.9 cm}|p{0.9 cm}|p{1.05 cm}|p{0.9 cm}|}
			\hline
%\rotatebox{90}{\textbf{Emotion } }&
%\rotatebox{90}{\textbf{Direction  ($rad$)}}&
%\rotatebox{90}{\textbf{Orientation ($rad$)} }&
%\rotatebox{90}{\textbf{Linear Velocity ($mm/s$) }}&
%\rotatebox{90}{\textbf{Angular Velocity ($rad/s$) }}&
%\rotatebox{90}{\textbf{Angle ($rad$)}}\\	
\textbf{Emotion}&\textbf{Direc-tion  ($rad$)} & \textbf{Orien-tation ($rad$)} & \textbf{Linear Velocity ($mm/s$) } & \textbf{Angular Velocity ($rad/s$) } & \textbf{Angle ($rad$)} \\
			\hline
			Happiness-1&$0$&$0$&$500$&$3$&$0.349$\\
			\hline
			\co Happiness-2&\co $0$&\co $0$&\co $900$&\co $3$&\co $0.174$\\
			\hline
			Anger-1&$\pi$&$0$&$500$&$3$&$0.087$\\
			\hline
			\co Anger-2&\co $0$&\co $0$&\co $900$&\co $1$&\co $0.087$\\
			\hline
			Fear-1&$\pi$&$\pi$&$900$&$2$&$0.174$\\
			\hline
			\co Fear-2&\co $\pi$&\co $\pi$&\co $500$&\co $2$&\co $0.087$\\
			\hline
			Sadness-1&$\pi$&$0$&$200$&$1$&$0.349$\\
			\hline
			\co Sadness-2&\co $0$&\co $\pi$&\co $200$&\co $1$&\co $0.349$\\
			\hline
			\end{tabular}
\end{table}


\subsection{Scene}

Stage discretization was used to give zones of movement instead of absolute positions. This idea was brought from human theatrical actors, who arrange their movements based on zones on the stage~\cite{wilson2009theatre}. This allows them to adapt their position based on other actors and stage dimensions. The stage was discretized in 9x9 matrix as shown in Figure~\ref{fig:stage_division}. Robot's movements are given in terms of matrix positions to the Emotional Enrichment System. The robot's final position is calculated by the Emotional Enrichment System during execution. In this setting, the scene was designed on a 3 x 3 meters stage, but in the presentation at the exhibition the stage was 2.5 x 2.5 meters.

\begin{figure}
	\centering
	\includegraphics[width=0.45\textwidth]{./Images/FourthCaseScene.png} 
	\caption{Stage discretization  used for the scene. The dark squares correspond to the each zone, while the numbers correspond to the ID given to each zone.}
	\label{fig:stage_division}
\end{figure} 

The scene can be described as follows. The robot starts in the middle of the stage to move to the upstage right (see~\cite{Musical}), close to the right wing. Then, the robot moves to upstage right center and rotates to $\pi/2$ left (see~\cite{Artopia}). Next the robot moves to the right center. Then it goes to the center. When it arrives there, it turns full back and move backwards to downstage center with a full front orientation. There, it turns full back to move to center. Finally the robot turns to profile right and it does a step back; then it goes to the upstage center and then upstage right. The sequence of robot's movements are depicted in Figure~\ref{fig:movement}.
\begin{figure*}
	\centering
	\includegraphics[width=0.95\textwidth]{./Images/fourthCaseSceneD.png} 
	\caption{Sequence of robot's movements. The red arrows show the trajectory, while the numbers show the order among the movements. a) The first ten movements b) The last five movements }
	\label{fig:movement}
\end{figure*}

Movements have been enriched by emotional expression as follows: movements one to five do not express any emotion, movements six to ten show fear. Movement eleven depicts happiness, and the remaining movements depict sadness. The actions describing this scene are executed by the Emotional Enrichment System and the emotion selection is designed via a graphical interface.

\subsection{Study}

This case study was done at an exhibition during a period of two days. People coming to the stand were asked to volunteer to participate in this study, which was divided in two parts. In the first one, each subject was exposed to two emotion expressions. %TODO Was it so? I have turned the sentence... 
The two emotions were generated to guarantee a random uniform presentation. In the second part, participants were explained that a small scene was going to be presented twice, and they had to select the one they liked more. The order of the scenes (e.g., with or without emotion) was randomly generated. The total number of volunteers was 256: 128 males, 126 females, and 2 that chose not to specify their gender. The average age was 27.29 years, with standard deviation of 16.58, minimum age was 4 and maximum 76. We accepted this variability as representative of a general population.
