Emotional Execution Tree is a connected acyclic graph $G(V,E)$ with $|V|$ vertexes and $|E|$ edges. The root and non-leaf nodes could be of either \textit{parallel} or \textit{sequential} type. The parallel node could be one out of four different sub-types: (i) action and emotion synchronous, (ii) action synchronous and emotion asynchronous, (iii) action asynchronous and emotion synchronous, or (iv) action and emotion asynchronous. Sequential nodes could just be one of two sub-types: emotion synchronous or asynchronous. Action synchronous means that each time that a parallel node receives a ``finish'' notification (i.e., success or failure), it will broadcast the message to all nodes that derived it and to its predecessor. If a sequence node receives a finish message, it will execute the next branch. When all branches have been executed, it communicates the end of the action. On the other hand, emotion synchronous means that each time that a node (either sequence or parallel) receives an emotion synchronization message, it will propagate the message to all branches to move to the consecutive emotional expression. If a node is principal and it has finished to execute all actions, it will notify its predecessor. 
 
This distinction creates the possibility to synchronize emotional changes without affecting the normal execution of an action and it also enables synchronization among parallel actions. Finally, leaf nodes could only be simple action nodes that have been implemented in the system. Any node can belong to one of two levels: principal or  secondary. If a node is principal, it will notify its predecessor about the messages that it has received, while the secondary node cannot propagate any message to its predecessor.%Why is this needed?
Compound actions are implemented combining all type of nodes. 