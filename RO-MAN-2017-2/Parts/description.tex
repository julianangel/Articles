The description files allow the parametrization of the system and its adaptation to different circumstances.  The system has the following parametric data:

\begin{itemize}

	\item \textit{Emotion description} gives information of the parameters that should be changed in all the simple actions to express the desired emotion. Therefore, for each combination emotion-action should be given a description of how the parameters are defined and how they change in time. 
%If there is any action that does not have any specification, the system will not change its parameters. 
The first version of this parameters is done for movement. Therefore, it was considered four descriptors for this type of parameter. The first is the reference, which inform the system the velocity used to calculate the parameter. This reference is used to let the modulation of the parameters depending on the character. The second descriptor is the tuple space and time. with this tuple is possible to calculate the velocity in which the movement must be done in order to convey the desire emotion. in order to enable the possibility to have different velocities during the execution, it could be described a sequence of this tuple. in this case, the space descriptor plays an important role, which is to inform the length in which the desired velocity must be used. This bring the last description which is the repetition that is used if there is more than one tuple and it should be repeat during the duration of the movement. The EBFN to described these descriptors in the system is the following: 
\begin{grammar}
[(colon){$\models$}]
[(semicolon)$|$]
[(comma){}]
[(period){\\}]
[(quote){\begin{bf}}{\end{bf}}]
[(nonterminal){$\langle$}{$\rangle$}]
<emotion description>: "'\{'","'emotion:'",<string> "','","'observation:'",<string> "','",<action>"'(,'",<action>)* "'\}'".
<action>:<string>"':\{'",<description> "','",<actions affected> "'\}'".
<description>:"'description:\{'",<action name>"','",<emotion>"','"\\
<parameter's type>"'\}'".
<action name>:"'emotionProfileAction:'",<string>.
<emotion>:"'emotionProfileEmotion:'",<string>.
<parameter's type>: "'movement\_parameter'".
<actions affected>:"'actions:\{'",<action parameter>,\\("','"<action parameter>)*"'\}'".
<action parameter>: <string>"':\{'",<reference>"','"\\<repetition>"','",<parameters>"'\}'".
<reference>: "'reference:'",<number>.
<repetition>: "'repetition:'","'yes'";"'no'".
<parameters>: "'parameters:['" <parameter description>\\("','",<parameter description>)* "']'".
<parameter description>: <movement parameter description>;<new parameters>.
<movement parameter description>:"'\{time:'",<number>"',space:'"\\<number>"'\}'".
\end{grammar}

	\item \textit{Character's emotions} parameters give the system information about character's ''rhythm'' for each pair action-emotion. To describe this changes, it was decided to used bias, amplitude and long as descriptors. Bias refers to a constant value that is added to all the sequence of emotional parameters. Amplitude amplify the velocity respect the bias if the number is bigger than one, or attenuate if it is between zero and one. 
%TODO if there is enough space add an imagine explaining these concepts


	\item \textit{Action message} contains all the necessary information for to execute an action. The following is the EBNF of an action:
\begin{grammar}
[(colon){$\models$}]
[(semicolon)$|$]
[(comma){}]
[(period){\\}]
[(quote){\begin{bf}}{\end{bf}}]
[(nonterminal){$\langle$}{$\rangle$}]
<action>: <simple action>;<compound action>;<context>.
<simple action>: "'\{'" <action header> "','","'parameters:['" <simple action parameters> "']\}'".
<compound action>: "'\{'" <action header> "','","'parameters:['" <compound action parameters> "']\}'".
<action header>: "'type:'", <action type> "','","'name: '" <string>  "','", <is primary>.
<simple action parameters>: "'\{'"<parameter header> "','" ,<parameter description> "'\}'". 
<compound action parameters>:  <simple action parameters> ("','", <simple action parameters>)* .
<context>:"'\{'" <action type> "','", <emotion sync> "','", <action sync> "','", <is primary> "','"\\" 'information:'", <string> "','"," 'actions: '", <action> "'\}'".
<parameter header>: "'type:'", <parameter type> "','"," 'name:'",<string>.
<parameter description>: <parameter amplitude>;<parameter circle>;<parameter landmark>;\\<parameter point>;<parameter speech>;\\<parameter square>;<parameter time>.
<parameter type>: "'mandatory\_parameter'";"'optional\_parameter'". 
<is primary>: "'yes'";"'no'".
<emotion sync>: "'yes'";"'no'".
<action type>: "'parallel\_context'";"'serial\_context'";"'simple\_action'";\\"'composite\_action'".
\end{grammar}
\end{itemize} 