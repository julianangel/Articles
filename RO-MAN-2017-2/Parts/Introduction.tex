The development of fast, cheap, and reliable electronics has enabled the creation of new devices and versatile robotic platforms. These new platforms' capabilities have expanded the frontiers of the robots applications to new environments where robot are expected to interact with humans, such as health care, and house cleaning, among others. However, bringing robots in these environment raises the challenge to increase robots' acceptance. Although this could be seen as an easy task that just would need improvements in robots' appearances and capabilities, it is possible that people would expect to treat robots as humans as they do with computers.~\cite{Reeves1996}, which makes necessary the creation of robots able to fulfil this expectations.

Some researchers have suggested that embedding emotion expression capabilities to robots could improve their acceptance in social environments~\cite{Pavia2014}. To this purpose, some researchers~\cite{Breazeal2002, Arras2012} have added specific emotional poses and expression to their robots. Others have studied how to convey emotions with specific platforms~\cite{Li2011, Brown2014}. Nevertheless, these works have created modules to show emotions that are strongly integrated to their solutions, which eliminate the possibility to re-use or adapt their systems into other projects.

In theory, the projection of emotion with humanoid embodiments could be simplified by mimicing the human movements. However, this idea cannot be fully implemented, due physical limitations in robots. Since an exact matching of human movements to convey emotions could not be used in robots \cite{Saerbeck2007,Canamero2010}, some researchers are studying diverse features and values to express emotions with different platforms. As a consequence, these results could not be widely used due to the difference among the platforms.

This paper presents an Emotional Enrichment System (EES) that modifies actions' parameters and adds additional actions to create the illusion of emotion expression in a robot. Although the EES was originally conceived to be used to facilitate the study of emotion projection, its design was devised to make it extendable to other platforms and adaptable to new tasks. To achieve this goal, the system relies on an Emotional Execution Tree (EXT), which is based on simple actions, sequential and parallel nodes. Additionally, the concept of compound actions is used to group a bunch of nodes, thus reducing the tree dimension and allowing the reuse of recurrent actions generated by specific combinations of simple actions and other nodes. This EXT has been formalized to give a guideline to further implementations and extensions.
 
The rest of the paper is organized as follows. Section~\ref{sec:related_work} provides a brief overview of relevant work. Section~\ref{sec:concepts} introduces all the concepts that are used in the system. Section~\ref{sec:emotional_execution_tree} gives the basic formalization of our system and principal components terms used on it. Section~\ref{sec:description} presents the EBFN created to specify the files used by the system. Section~\ref{sec:implementation} describes the implementation of the system and shows two demonstrations done with the system using platforms with different capabilities.
