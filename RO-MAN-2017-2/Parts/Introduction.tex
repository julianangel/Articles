The development of fast, cheap, and reliable electronics has enabled the creation of new devices and versatile robotic platforms. These new platforms' capabilities have expanded the frontiers of the robots applications  to new environments where robot are expected to interact with humans, such as health care, and house cleaning, among others. However, bringing robots in these environment raises the challenge to increase robots' acceptance. Although this could be seen as an easy task that just would need improvements in robots' appearances and capabilities, it is possible that people would expect to treat robots as humans has  as they do with computers.~\cite{Reeves1996}, which makes necessary the creation of robots that fulfil this expectations.

Some researchers have suggested that embedding emotion expression capabilities to robots could improve their acceptance in social environments~\cite{Pavia2014}. As consequence researchers~\cite{Breazeal2002,Arras2012} have added specific emotional poses and expression to their robots. Others have studied how to convey emotions with specific platforms~\cite{Li2011,Brown2014}. Nevertheless, these works have created modules to show emotions that are strongly integrated to their solutions, which eliminate the possibility to re-use or adapt their systems into other projects.

In theory, the projection of emotion with humanoid embodiments could be simplified to mimic the same movements that humans. However, this idea is not possible due robots physical limitations. Therefore, an exact matching of humans' movements to convey emotions could not be used in robots \cite{Saerbeck2007,Canamero2010}. Therefore diverse researchers are studying diverse features and values to express emotions with different platforms. As a consequence, these results could not be widely used due to the differences of the platforms.

This paper presents an Emotional Enrichment System (EES), which modify actions' parameters and add additional actions to create the illusion of emotion expression in a robot. Although the EES was originally conceived to be used to facilitate the study of emotion projection, its design was devised to make it extendable to other platforms and adaptable to new tasks. To achieve this goal, the system relies on an Emotional Execution Tree (EXT), which is based on simple actions, sequential and parallel nodes. Additionally, it is used the concept of compound actions to group a bunch of nodes, which reduces the tree dimension and allows the reuse of recurrent actions  generated by specific combinations of simple actions and other nodes. This EXT has been formalized to give a guideline to further implementations and extensions.
 
The rest of the paper is organized as follows. Section~\ref{sec:related_work} provides a brief overview of particularly relevant work related to our system. Section~\ref{ref:theatre} gives a brief explanation on TheatreBot architecture. Section~\ref{ref:general_system} gives a general introduction to the system ideas. Section~\ref{sec:emotional_execution_tree} gives the basic formalization of our system and principal components terms used on it. Section~\ref{sec:implementation} describes the implementation of the system and shows two demonstrations done with the system using platforms with different capabilities.