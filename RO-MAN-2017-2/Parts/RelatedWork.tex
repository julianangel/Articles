The use of emotion enrichment to improve human robot interaction is not new trend. There have been several works that have enhance their social robots with emotions or study how to convey them in robotic platforms~\cite{Li2011,Brown2014}. One of the most well-known expressive robots is Kismet~\cite{Breazeal2002}, a robotic face able to interact with people and show emotions. This platform uses a specific set of movements based on the Ekman's studies on human emotion expression~\cite{Ekman2004}. Other approaches have tried to use anthropomorphic~\cite{Arras2012} and human-like platforms to convey emotions to study the response of people towards the robot. However, the emotions portrayed were hand-coded, hard-wired to the respective platforms, and their parametrization was not available.\\
On the other hand, studies focused on entertainment robotics have tried to introduce emotional actions to improve the audience's experience. Breazeal and collaborators~\cite{Breazeal2003} used one robot on the stage. This anemone-like robot had few behaviours, which included getting scared when a person comes too close; the robot was able to show some basic emotions (i.e., fear and interest). Knight~\cite{Knight2011b,Knight2010} used the platform NAO to produce a sort of stand-up comedy. The robot performs basic actions to add some expressiveness to the joke, but it is not intended to project any emotion. Trying to add some theatrical realism, Breazeal and collaborators~\cite{Breazeal2008} designed and implemented a system to control a lamp. The main characteristic of this lamp is that it could be controlled by just one person, which selected pre-coded emotions.\\
Other works in performance robotics have developed systems that do not convey any kind of emotion as \textit{Roboscopie}~\cite{Roboscopie2012,Lemaignan2012}, Fan and collaborators~\cite{Fan2009,Fan2013}, and adaptation of Shakespeare's  Midsummer Night's Dream~\cite{murphy2011} uses robots in performances with real actors but without any emotion expression.\\
Although these works use emotions, their main focus was the interpretation of postures or just the use of emotions to increase their robot appealing, but none of them considered the importance of emotional system that could be used by others. As a consequence, most of these works have created emotional systems that could just work in their specific system.