%%%%%%%%%%%%%%%%%%%%%%%%%%%%%%%%%%%%%%%%%%%%%%%%%%%%%%%%%%%%%%%%%%%%%%%%%%%%%%%%
%2345678901234567890123456789012345678901234567890123456789012345678901234567890
%        1         2         3         4         5         6         7         8

\documentclass[letterpaper, 10 pt, conference]{ieeeconf}  % Comment this line out if you need a4paper

%\documentclass[a4paper, 10pt, conference]{ieeeconf}      % Use this line for a4 paper
                                   % Needed to meet printer requirements.

% See the \addtolength command later in the file to balance the column lengths
% on the last page of the document

% The following packages can be found on http:\\www.ctan.org
\usepackage{graphics} % for pdf, bitmapped graphics files
\usepackage{epsfig} % for postscript graphics files
\usepackage{multirow} 
\usepackage{subcaption}
\usepackage{caption}
\usepackage{url}
\usepackage{bnf}
\usepackage{slashbox}
\newcommand{\co}{\cellcolor{gray!40}}
\usepackage[table]{xcolor}
%\usepackage{mathptmx} % assumes new font selection scheme installed
%\usepackage{times} % assumes new font selection scheme installed
%\usepackage{amsmath} % assumes amsmath package installed
%\usepackage{amssymb}  % assumes amsmath package installed
\IEEEoverridecommandlockouts                              % This command is only needed if 
                                                          % you want to use the \thanks command

\overrideIEEEmargins   

\title{\LARGE \bf
Enriching Robot's Actions with Affective Movements
}


\author{Julian M. Angel-Fernandez$^{1}$ and Andrea Bonarini$^{2}$% <-this % stops a space
\thanks{$^{1}$Julian M. Angel-Fernandez is a Research at Automation and Control Institute, Vienna University of Technology, Vienna, Austria
        {\tt\small julian.angel.fernandez@tuwien.ac.at}}%
\thanks{$^{2}$Andrea Bonarini is a Professor at the Department of Electronics, Information, and Bioengineering, Politecnico di Milano, Milan, Italy,
        {\tt\small andrea.bonarini@polimi.it}}%
}


\begin{document}



\maketitle
\thispagestyle{empty}
\pagestyle{empty}


%%%%%%%%%%%%%%%%%%%%%%%%%%%%%%%%%%%%%%%%%%%%%%%%%%%%%%%%%%%%%%%%%%%%%%%%%%%%%%%%
\begin{abstract}
Emotions are considered by many researches as a characteristic that could be beneficial in social robotics, since they enrich human-robot interaction with non-verbal clues. Although there have been works that have studied emotion expression in robots, mechanism to generate emotion are highly integrated with the rest of the system. This unable the possibility to use their approaches in other applications. This paper present a system that has been initially created to facilitate the study of emotion projection, but it was designed to enable its adaptations in other fields. The emotional enrichment system has been envisioned to be used with any action decision system. A description of the system components and its characteristics are provided. The system has been adapted to two different platforms with different grades of freedom: Keepon and Triskarino. 
\end{abstract}

\section{Introduction}
Social environments are intricate spaces where people behave in ways that aim to their acceptance in social groups. Similarly, studies in social robotics have shown that robots acceptance increases when robots project a high social presence~\cite{Heerink08}. The straightest way to fulfill is through imitation of humans' characteristics, such us body form, behaviors and social characteristics. This has inspired most researchers to focus on how express emotions and mental states exploiting anthropomorphic features, in many occasions relying just on faces. However emotions and mental states are not just presented through facial expression, but also by body postures and other features~\cite{Gelder2008}. Nonetheless in psychology has been a clear tendency to study the role of human face in emotion projection~\cite{Ekman2004},~\cite{kleinsmith2012affective} and mental states. This trend has been followed by robotics community, where anthropomorphic faces (e.g.,~\cite{Arras2012},~\cite{Breazeal2002}) and bodies (e.g.,~\cite{Canamero2010},~\cite{haering2011},~\cite{Destephe2013}) have been widely used to convey emotions.
Nevertheless, in many situations the presence of anthropomorphic elements would be out of place and not justified by the main robot's functionalities. Most of the current and future robotics platforms on the market will not require anthropomorphic faces or limbs. In some cases, like, for instance, in floor cleaning robots, anthropomorphic characteristics could even be detrimental to robot's task accomplishment.
This generates the necessity to study other mechanisms that could help to project emotions, which could give people an idea about the robots' state, and engage the user in long term relations.

The amount of works studying non-anthropomorphic features in robotics (e.g.,~\cite{Saerbeck2010,Lakatos2014,Sharma2013,Novika2015}) remains still small in comparison to those that exploit anthropomorphic features. Moreover, these works do not prescribe any specific range of values for the characteristics to be used to express the  implemented emotions. For example, Suk and collaborators~\cite{NAM2014}, in their study, give specific values for acceleration, curvature, but their connection to specific emotions is not given. Rather, they gave a relationship between their features and values in terms of valence and arousal. Another possibility is the use of professional human actors to study how they convey certain emotions. However, a direct mapping between humans and robots is not possible~\cite{Saerbeck2007,Canamero2010} due to robots' physical capabilities. In addition, the significance of the agreement obtained from the studies that used human actors is still on discussion~\cite{Russell2003}. As a consequence is required the necessity to determine to project emotions in non-anthropomorphic platforms.
 
In order to get a better understanding of other features and values that could be used to convey specific emotions, this paper presents an experiment that was designed to identify specific values for some movement features that could be used to express the following emotions, selected among the ones suggested by Ekman as basic~\cite{Ekman2004}: happiness, anger, fear, and sadness. The considered features are: oscillation angle, linear and angular velocity, direction and orientation, identified as independent variables. The perceived emotions and their intensities were considered as dependent variables. It is to be observed that it was given the possibility to the subjects to provide their evaluation of emotional intensity for more than one emotion for each movement of the robot or treatment. The experiment took place in Politecnico di Milano during June and July of 2015, where students from diverse departments were asked to participate without any economical retribution. Krippendorff's alpha agreement~\cite{Krippendorff2007} ($\alpha$) was used to evaluate the agreement among the participants for each treatment. For each of the four emotions was generated a top 10 list based on alpha agreement and perceived intensity. The results suggest that fear is perceived when the robot is looking at the subjects while moving far from them fast. Sadness is associated to slow velocities with slow angular velocity and small oscillation angle. Anger is attributed to fast velocities, both angular and linear, small angle of oscillation and the robot facing the subjects while approaching them.

This paper is organized as follows. The next section introduces previous studies done in human emotion and robot emotions projection. Section~\ref{sec:system} introduces the robotic platform and the software used in the experiment. The experiment's design is explained in section~\ref{sec:experiment}. Finally sections~\ref{sec:experiment} and~\ref{sec:result} present the study done and the obtained results.
\section{Related Work}
The use of emotion enrichment to improve human robot interaction is not new. There have been several researchers that have enhanced their social robots with emotions or studied how to convey emotions in robotic platforms~\cite{Li2011,Brown2014}. One of the first, well-known expressive robots is Kismet~\cite{Breazeal2002}, a robotic face able to interact with people and to show emotions. This platform uses a specific set of movements based on the Ekman's studies on human emotion expression~\cite{Ekman2004}. Other approaches have tried to use anthropomorphic~\cite{Arras2012} and human-like platforms to convey emotions to study the response of people towards the robot. However, the emotions portrayed were hand-coded, hard-wired to the respective platforms, and their parametrization is not available.

On the other hand, studies focused on entertainment robotics have tried to introduce emotional actions to improve the audience's experience. Breazeal and collaborators~\cite{Breazeal2003} used one robot on the stage. This anemone-like robot had few behaviours, which included getting scared when a person comes too close; the robot was able to show some basic emotions (i.e., fear and interest). Knight~\cite{Knight2011b,Knight2010} used the platform NAO to produce a sort of stand-up comedy. The robot performs basic actions to add some expressiveness to the joke, but it is not intended to project any emotion. Trying to add some theatrical realism, Breazeal and collaborators~\cite{Breazeal2008} designed and implemented a system to control a lamp. The main characteristic of this lamp is that it could be controlled by just one person, by selecting pre-coded emotions.

Other works in performance robotics have developed systems that do not convey any kind of emotion as \textit{Roboscopie}~\cite{Roboscopie2012,Lemaignan2012}, Fan and collaborators~\cite{Fan2009,Fan2013}, and adaptation of Shakespeare's  Midsummer Night's Dream~\cite{murphy2011} use robots in performances with real actors, but without any emotional expression.\\
Although these works use emotions, their main focus was the interpretation of postures or just the use of emotions to increase their robot appealing, but none of them considered the importance of an emotional system that could be used by others. As a consequence, most of these works have created emotional systems that could just work in their specific framework.

\section{Basic Concepts}
\label{sec:concepts}
The system is based on six main concepts: \textit{simple actions}, \textit{compound actions}, \textit{action message}, \textit{emotional descriptors}, \textit{character description} and \textit{emotional execution tree}. \textit{Action message} establishes the structure of the message to describe any kind of action (i.e., simple and compound). This message also specifies how the actions are executed (i.e., in parallel or in sequence) and which action is predominant (i.e., primary or secondary). \textit{Emotional parameters} describe how the emotional enrichment should be done to convey a specific emotion in a specific simple action. This description could also include addition of other simple actions and vary over time. \textit{Character description} enables the possibility to establish how to modify emotional expressions to generate diverse treats. Finally, \textit{Emotional Execution Tree} is a computational representation of desired actions that should be executed. This tree is first created from the action message description and then modified using the emotional parameters and character description.  
\subsection{Simple and Compound Actions}
In order to generate a system that could be used in diverse platforms, an abstraction level in which all of them could fall is required. Therefore simple and compound actions are used to achieve this goal. Simple actions are actions that are considered as primitives: they are used as building blocks. Therefore, these actions are described in the system and are the ones in which the emotional enrichment takes place. Their description specifies mandatory and optional parameters that are required to execute an action. Compound actions are actions that are created from simple actions. These actions are not implemented in the system, but, if it is needed, they can be described in it (e.g., compound actions that are used often).

The simple actions to be implemented to test the system were selected by considering platforms' capabilities and the requirements. The eight actions selected are: move body, oscillate body, move shoulder, oscillate shoulder, move torso, oscillate torso, and do nothing. Description for each action, its mandatory parameters and optional parameters are shown in Table~\ref{table:actions_implemented}.

\begin{table}
\centering
\caption{Description of the seven simple actions implemented, and their respective parameters. Where  P is 2D position, V is 2D velocity vector and angular velocity, and T is time}
\label{table:actions_implemented}
\begin{tabular}{|c|p{3.9cm}|p{1.4cm}|}
\hline
\textbf{Action Name}& \textbf{Description} &\textbf{Parameter(s)} \\
\hline
Do nothing & It waits for a time $t$ before it is terminated. It could be seen as a delay.  & $T$\\
\hline
Move body & It moves the platform from its current position $a$ to a desired position $b$. & $P$, and $V$\\
\hline
Oscillate body & It generates an oscillation in the whole platform by an angle $\theta$. &  $\theta$ and $V$ \\
\hline
Move shoulder & It moves the shoulders to a desired angle $\theta$. It is considered as angular movement. & $\theta$ and $V$ \\
\hline
Oscillate shoulder & It oscillates the shoulders by a given angle $\theta$ & $\theta$ and $V$\\
\hline
Move torso & It moves the torso to a desired angle in $yaw$, $pitch$ and $roll$& $yaw$, $pithc$, $roll$ and $V$\\
\hline
Oscillate torso & It oscillates the shoulders by a given angle $\theta$ & $\theta$ and $V$\\  
\hline
\end{tabular}
\end{table}

%TODO add an example for this, very simple and presenting how to do it. IT WOULD BE VERY NICE
 
\section{Describing Components}
The description files allow the parametrization of the system and its adaptation to different circumstances.  The system has the following parametric data:

\begin{itemize}

	\item \textit{Emotion description} gives information of the parameters that should be changed in all the simple actions to express the desired emotion. Therefore, for each combination emotion-action should be given a description of how the parameters are defined and how they change in time. 
%If there is any action that does not have any specification, the system will not change its parameters. 
The first version of this parameters is done for movement. Therefore, it was considered four descriptors for this type of parameter. The first is the reference, which inform the system the velocity used to calculate the parameter. This reference is used to let the modulation of the parameters depending on the character. The second descriptor is the tuple space and time. with this tuple is possible to calculate the velocity in which the movement must be done in order to convey the desire emotion. in order to enable the possibility to have different velocities during the execution, it could be described a sequence of this tuple. in this case, the space descriptor plays an important role, which is to inform the length in which the desired velocity must be used. This bring the last description which is the repetition that is used if there is more than one tuple and it should be repeat during the duration of the movement. The EBFN to described these descriptors in the system is the following: 
\begin{grammar}
[(colon){$\models$}]
[(semicolon)$|$]
[(comma){}]
[(period){\\}]
[(quote){\begin{bf}}{\end{bf}}]
[(nonterminal){$\langle$}{$\rangle$}]
<emotion description>: "'\{'","'emotion:'",<string> "','","'observation:'",<string> "','",<action>"'(,'",<action>)* "'\}'".
<action>:<string>"':\{'",<description> "','",<actions affected> "'\}'".
<description>:"'description:\{'",<action name>"','",<emotion>"','"\\
<parameter's type>"'\}'".
<action name>:"'emotionProfileAction:'",<string>.
<emotion>:"'emotionProfileEmotion:'",<string>.
<parameter's type>: "'movement\_parameter'".
<actions affected>:"'actions:\{'",<action parameter>,\\("','"<action parameter>)*"'\}'".
<action parameter>: <string>"':\{'",<reference>"','"\\<repetition>"','",<parameters>"'\}'".
<reference>: "'reference:'",<number>.
<repetition>: "'repetition:'","'yes'";"'no'".
<parameters>: "'parameters:['" <parameter description>\\("','",<parameter description>)* "']'".
<parameter description>: <movement parameter description>;<new parameters>.
<movement parameter description>:"'\{time:'",<number>"',space:'"\\<number>"'\}'".
\end{grammar}

	\item \textit{Character's emotions} parameters give the system information about character's ''rhythm'' for each pair action-emotion. To describe this changes, it was decided to used bias, amplitude and long as descriptors. Bias refers to a constant value that is added to all the sequence of emotional parameters. Amplitude amplify the velocity respect the bias if the number is bigger than one, or attenuate if it is between zero and one. 
%TODO if there is enough space add an imagine explaining these concepts


	\item \textit{Action message} contains all the necessary information for to execute an action. The following is the EBNF of an action:
\begin{grammar}
[(colon){$\models$}]
[(semicolon)$|$]
[(comma){}]
[(period){\\}]
[(quote){\begin{bf}}{\end{bf}}]
[(nonterminal){$\langle$}{$\rangle$}]
<action>: <simple action>;<compound action>;<context>.
<simple action>: "'\{'" <action header> "','","'parameters:['" <simple action parameters> "']\}'".
<compound action>: "'\{'" <action header> "','","'parameters:['" <compound action parameters> "']\}'".
<action header>: "'type:'", <action type> "','","'name: '" <string>  "','", <is primary>.
<simple action parameters>: "'\{'"<parameter header> "','" ,<parameter description> "'\}'". 
<compound action parameters>:  <simple action parameters> ("','", <simple action parameters>)* .
<context>:"'\{'" <action type> "','", <emotion sync> "','", <action sync> "','", <is primary> "','"\\" 'information:'", <string> "','"," 'actions: '", <action> "'\}'".
<parameter header>: "'type:'", <parameter type> "','"," 'name:'",<string>.
<parameter description>: <parameter amplitude>;<parameter circle>;<parameter landmark>;\\<parameter point>;<parameter speech>;\\<parameter square>;<parameter time>.
<parameter type>: "'mandatory\_parameter'";"'optional\_parameter'". 
<is primary>: "'yes'";"'no'".
<emotion sync>: "'yes'";"'no'".
<action type>: "'parallel\_context'";"'serial\_context'";"'simple\_action'";\\"'composite\_action'".
\end{grammar}
\end{itemize} 
\section{Emotional Tree}
Emotional Execution Tree is a connected acyclic graph $G(V,E)$ with $|V|$ vertexes and $|E|$ edges. The root and non-leaf nodes could be of either \textit{parallel} or \textit{sequential} type. The parallel node could be one out of four different sub-types: (i) action and emotion synchronous, (ii) action synchronous and emotion asynchronous, (iii) action asynchronous and emotion synchronous, or (iv) action and emotion asynchronous. Sequential nodes could just be one of two sub-types: emotion synchronous or asynchronous. Action synchronous means that each time that a parallel node receives a ``finish'' notification (i.e., success or failure), it will broadcast the message to all nodes that derived it and to its predecessor. If a sequence node receives a finish message, it will execute the next branch. When all branches have been executed, it communicates the end of the action. On the other hand, emotion synchronous means that each time that a node (either sequence or parallel) receives an emotion synchronization message, it will propagate the message to all branches to move to the consecutive emotional expression. If a node is principal and it has finished to execute all actions, it will notify its predecessor. 
 
This distinction creates the possibility to synchronize emotional changes without affecting the normal execution of an action and it also enables synchronization among parallel actions. Finally, leaf nodes could only be simple action nodes that have been implemented in the system. Any node can belong to one of two levels: principal or  secondary. If a node is principal, it will notify its predecessor about the messages that it has received, while the secondary node cannot propagate any message to its predecessor.%Why is this needed?
Compound actions are implemented combining all type of nodes. 
\section{Implementation}
The system was implemented in \textit{C++} and interfaced with ROS. The design (Fig.~\ref{fig:system_architecture}) was created following the description presented in the previous sections. 
\begin{figure*}
	\centering
	\includegraphics[width=1.0\textwidth]{Images/SystemArchitecture.png} 	
	\caption{General system design. Each simple action corresponds to one ROS node, and there is just one node for the emotion enrichment system. The ovals represent the ROS topic parameters, rectangles represent \textit{black boxes}, and texts outside containers represent input files that contain the system parametrization.}
	\label{fig:system_architecture}
\end{figure*}
The emotion enrichment core is divided in three different modules. 
Each module is responsible for one of the following phases:
\begin{enumerate}
	\item \textit{Generation of emotional execution tree:} this phase starts every time that a new action message is received. The process begins by parsing the format, verifying that the actions described on it exist in the system, and that the parameters correspond to the ones expected by each action included in the message. 
When the verification is done, and all the actions exist and the parameters correspond, an EXT such as the one presented in Figure~\ref{fig:emotional_enrichment} is created.
	\item \textit{Emotion addition:} uses the EXT created in the previous phase. In this phase new simple actions ($sa$)
can be added to the EXT, and the $sa$'s parameters are modified following the emotion description, which is loaded from files. This process is broken down in two steps. First, all the actions that are required to convey the desired emotion, and that are not yet present are added. Second, the emotional parameters are modulated basing on the emotion's intensity and character traits. The final EXT is presented in Figure~\ref{fig:reference}.
	\begin{figure}
		\centering
		%Representation
	\includegraphics[width=0.45\textwidth]{./Images/exampleTreeE.png}
	\caption{EXT enriched with emotion for the EXT presented in Figure~\ref{fig:emotional_enrichment}.} 
	\label{fig:reference}
	\end{figure}
	\item \textit{Execution:} this is the last phase and it is done after the EXT is ''coloured'' with emotional characteristics (actions additions and emotional parameters). The decision to have two different communication channels, one for action parameters and another for the action emotional parameters, was taken to enable the possibility to update the emotional parameters without interfering with the current execution. 
\end{enumerate}

All the text message broadcast among the nodes are written using JSON format, which is human understandable, light, and can be parsed by many systems.\\ 
To test the system, two different platforms were used: Keepon Pro %TODO da citare la prima volta il paper descrittivo
and an own made platform called Triskarino.
Keepon has been used to test the interoperability of the system to different platforms, while Triskarino has been used in diverse case studies to explore emotion projection with a non anthropomorphic platform.
%%%%%%%%%%%%%%%%%%%%%%%%%
%%%%%%%%%%%%%%%%%%%%%%%%%%%%%
\subsection{Keepon Test}
To test the system with Keepon (Fig.~\ref{fig:keepon}) was just necessary to implement the platform's controllers in each one of the simple action ROS nodes. Given that this platform does not have the capability to displace its body, the action move body was not implemented. Once added these controllers to the system, we proceeded to modify the configuration files to use this platform instead of Triskarino. With this small modification, we were able to change from one platform to other. In the video\footnote{YouTube video name Emotional Enrichment System, url: https://youtu.be/bRSXQ0rzkO8}
is shown the test where Keepon had to move its torso forward and backward to express a ``happy'' emotion. 
\begin{figure}
	\centering
	\includegraphics[width=0.2\textwidth]{./Images/Keepon.jpg}
	\caption{Keepon platform.}
	\label{fig:keepon}
\end{figure} 
The action is given by console telling the robot to bend the torso to a desired angle in x. The torso oscillation in y is added automatically by the system following the description given to happiness. 
 %%%%%%%%%%%%%%%%%%%%%
%%%%%%%%%%%%%%%%%%%%%%%%%%%%%
\subsection{Triskarino Test}
The system has been widely used with Triskarino (Fig.~\ref{fig:robot}) during our studies on projecting emotions with a non human like platform. 
\begin{figure}
	\centering
	\begin{subfigure}[c]{0.2\textwidth}
	\includegraphics[width=\textwidth]{./Images/upperFourthD.png}
	\caption{Triskarino design.}
	\label{fig:design}
	\end{subfigure}
	\begin{subfigure}[c]{0.2\textwidth}
	\includegraphics[width=\textwidth]{./Images/platform_fome.jpg}
	\caption{Triskarino with its cover.}
	\label{fig:triskar}
	\end{subfigure}
	\caption{Triskarino platform.}
	\label{fig:robot}
\end{figure}  
 During these studies the system was used in different experiments to move the robot on a straight line showing different emotions, each one selected from our command console. Moreover, to show the whole capabilities of the emotional enrichment system, a little scene was set up. The scene is a modification of the first part of the balcony scene of Shakespeare's Romeo and Juliet play~\cite{RAndJ}. The simplified sketch of our version could be seen in the Fig.~\ref{fig:triskar_test}. As it can be seen the stage was divided in 81 squares and the positions were given to the system in terms of the desire square to be reached, which allows the robot to adjust to the stage's dimension.
The whole sequence of actions were specified as unique action, having several sequential nodes and just move body action. The other actions, such as oscillate body or blend upper body were added online accordingly the desired emotion.  


\section{Conclusions and Further work}
An Emotional Enrichment System has been designed and implemented to enrich robots' movements with emotions. To achieve this, it was used an Emotional Execution Tree created from three different types of nodes: simple actions, parallel, and sequential. Simple actions are functions that map a set of parameters to specific movements. Sequential executes in order the sequence of actions associated to this type of node, while parallel executes them all at the same time. To enable synchronization among simple actions, parallel nodes could be one of four different subtypes, and sequential just one of two different subtypes. A formalization of the Emotional Execution Tree and the principal consideration during the implementation of the system have also been provided. To show the system's versatility, it was used with quite different platforms such as a Keepon Pro and Triskarino. Keepon was used to perform simple actions, while Triskarino was used to test complex actions with different parametrizations of emotions.\\
The results obtained show that the system could enrich the robotic actions with emotions, which could be parametrized from configuration files. Additionally, the design of the system makes it possible to  adopt it for different platforms using the same action description.
%The case study presented in this paper was done to cross validate the findings in the experiment and verify whether the participants would prefer scenes when the robot expresses emotions or rather moves without any emotion expression. For each one of four emotions (i.e. Anger, Happiness, Sadness and Fear) studied in the experiment were selected a two set of parameters. These parameters were described in files and used as input to the Enrichment Emotional System, which was in charge to combine actions with emotions. The results show that both implementations of happiness were confused with anger and excitement, while one implementation of anger was just confused with excitement. Both implementations of sadness were confused with tenderness and fear. Both implementations of fear had a recognition rate over 50\%. Scene's results show that people prefer scenes with emotional movements and there is not any difference in gender.

Additionally to the results already mentioned, there are words that could bias participants' perception. For instance happiness and anger were considered as excitement. This misinterpretation should not be a surprise given the fact that there is not a unique definition of emotion~\cite{Plutchik2001,cacioppo2000handbook}, and each person would interpret a situation differently, so they will give a different label to the presented movement. Moreover a misinterpretation of Happiness and Anger could suggest that additional features (e.g. trajectory or shape) should be added to increase differentiation between these two emotions. For example, Venture and collaborators~\cite{Venture2014} had found out that in human bodies the recognition rate of anger and fear are increased when torso and head are downwards. On the other hand, they found that happiness perception is increased when the torso and head are move upwards. This example could bring some insight to possible body changes that could occur in non-human like bodies, but it should tested in this kind of platforms to confirm if the same impact is reached.
 
\bibliographystyle{IEEEtran}
\bibliography{Bibliography,BibloNew,Biblography}

\addtolength{\textheight}{-12cm}
\end{document}
